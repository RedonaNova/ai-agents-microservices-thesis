\section{Ижил төстэй системүүдийн судалгаа}

Хиймэл оюун агентийг нэвтрүүлсэн хэд хэдэн үйлдвэрийн түвшний шийдлүүд байдаг. Эдгээр системүүдтэй харьцуулалт хийснээр энэ судалгааны санал болгож буй зохиомжийн онцлог байр суурийг тодорхойлно.  ~\cite{camunda} ~\cite{camunda} ~\cite{bpmnEngine}.

\begin{longtable}{p{0.17\textwidth} p{0.25\textwidth} p{0.25\textwidth} p{0.25\textwidth}}
\caption{Ижил төстэй системүүдийн харьцуулалт} \label{tab:similar-systems-comparison} \\
\textbf{Шинж чанар} & \textbf{Inngest} & \textbf{Temporal} & \textbf{Camunda} \\
\hline
\endfirsthead
\caption[]{Ижил төстэй системүүдийн харьцуулалт (үргэлжлэл)} \\
\textbf{Шинж чанар} & \textbf{Inngest} & \textbf{Temporal} & \textbf{Camunda} \\
\hline
\endhead
Төрөл & Serverless workflow платформ & Durable execution платформ & BPMN процессын автоматжуулалт \\
\hline
Архитектур & EDA, хэрэглэгчийн функцийн түвшинд & Дэд процессийн түвшинтэй, олон хэлний дэмжлэг & BPMN 2.0 стандарт, enterprise \\
\hline
Хиймэл оюун дэмжлэг &  LLM интеграци байдаг ч, RAG дутмаг байна & Тодорхой процессийг л гүйцэтгэх тул, өөрийгөө удирдах агент болч чадахгүй & Байхгүй \\
\hline
Байршуулалт & Serverless (вендороор) & Нарийн төвөгтэй (олон бүрэлдэхүүн) & Хүнд \\
\hline
Урсгалын боловсруулалт & Хязгаарлалттай & Хязгаарлалттай & Хязгаарлалттай (BPMN-ын цар хүрээнд хязгаарлагдмал) \\
\hline
Ашиглагчид & Startup, жижиг баг & Uber, Netflix, Stripe & Томоохон байгууллагууд \\
\hline
\end{longtable}

\subsection{Бидний санал болгож буй загварын ялгаа}

Дээрх системүүдээс ялгаатай нь бидний загвар нь дараах онцлогуудтай.

\begin{longtable}{p{0.28\textwidth} p{0.65\textwidth}}
\caption{Бидний санал болгож буй загварын онцлог} \label{tab:our-model-features} \\
\textbf{Онцлог} & \textbf{Тайлбар} \\
\hline
\endfirsthead
\caption[]{Бидний санал болгож буй загварын онцлог (үргэлжлэл)} \\
\textbf{Онцлог} & \textbf{Тайлбар} \\
\hline
\endhead
Хиймэл оюун агентуудын интеграц & LLM, RAG, зааврын инженерчлэл, агентын төлөвлөгч зэргийг дотоод сиситемдээ интеграц найдвартай байдлаар интеграц хийж болдог. \\
\hline
Kafka-Flink суурилсан EDA & Өгөгдлийн бодит цагийн боловсруулалт, хэвтээ өргөжүүлэлтийн дэмжлэг, үйл явдлын хадгалалт хийснээр дахин ачааллуулж, үнэлэх боломжтой \\
\hline
Ухаалаг зохион байгуулалт & Агент бүр хоорондоо өөр өөр зориулалттай мэдээллүүд солилцох ба маш нарийн даалгавруудыг гүйцэтгэж чадна \\
\hline
Нээлттэй эх & Kafka, Flink, PostgreSQL, Redis зэрэг бүрэн нээлттэй эх тул өөрийн бизнесийн онцлогоор хөгжүүлэлт хийх боломжтой \\
\hline
Мэдээллийн аюулгүй байдал & Суурь модел нь гаралтад суурилсан тул гарч болох эмзэг мэдээллийг хэрэглэгчдэд хүргэхээс өмнө хааж болно \\
\hline
Өргөтгөл & Шинэ агентууд нэмж, зохион байгуулагч ба RAG-д уг агентийн чадамжийг бүртгэж хэрэглээнд амар байдлаар нэвтэрнэ \\

\hline
\end{longtable}

