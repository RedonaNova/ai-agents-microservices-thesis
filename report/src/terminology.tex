\chapter*{Нэр томьёоны тайлбар}
\addcontentsline{toc}{chapter}{Нэр томьёоны тайлбар}
\begin{description}
\item[AI Engineering] Хиймэл оюуны инженерчлэл
\item[Foundation Model] Суурь загвар
\item[Language Model] Хэлний загвар
\item[Masked Language Model] Далдлагдсан хэлний загвар
\item[Large Language Model] Том хэлний загвар
\item[Labeled data] Тэмдэглэгдсэн өгөгдөл
\item[Retrieval-Augmented Generation] Хайлтаар нэмэгдүүлсэн үүсгэлт
\item[Retrieval algorithm] Хайлтын алгоритм
\item[Prompt Engineering] Зааврын инженерчлэл
\item[Machine Learning] Машин сургалт
\item[Deep Learning] Гүн сургалт
\item[Microservices Architecture] Микросервис архитектур
\item[Event-Driven Architecture] Үйл явдлаар удирдагдах архитектур
\item[Message Queue] Мессежийн дараалал
\item[Vector Database] Векторын өгөгдлийн сан
\item[Embedding] Векторчилсон хэсгийн төлөөлөл
\item[Token] Токен (хэлний загвард тэмдэгт мөрийн таслалын нэгж)
\item[Inference] Гаргалгаа (загвараас үр дүн гаргах үйл явц)
\item[Fine-tuning] Нарийвчилсан сургалт
\item[Pre-training] Урьдчилсан сургалт
\item[Supervised Learning] Удирдлагатай сургалт
\item[Self-supervised Learning] Өөрийгөө удирдсан сургалт
\item[Reinforcement Learning] Бэхжүүлсэн сургалт
\item[Temperature] Температур (загварын бүтээлч чанарын параметр)
\item[Hallucination] Төөрөгдөл (загварын буруу мэдээлэл үүсгэх үзэгдэл)
\item[Context Window] Контекстийн цонх
\item[Agent] Агент (бие даан үйлдэл хийх систем)
\item[Tool] Хэрэглүүр (хиймэл оюуны ашиглах хэрэглүүрүүд, ихэвчлэн API хэлбэрээр дамждаг.)
\item[Orchestrator] Зохион байгуулагч
\item[Consumer] Хэрэглэгч (мессеж хүлээн авагч)
\item[Producer] Үйлдвэрлэгч (мессеж илгээгч)
\item[Topic] Сэдэв (Kafka-гийн мессежийн ангилал)
\item[Partition] Хэсэглэл
\item[Stream Processing] Урсгал боловсруулалт
\item[Latency] Хоцрогдол
\item[Throughput] Дамжуулалт
\item[Scalability] Өргөжих чадвар
\item[Resilience] Уян хатан чанар байдал
\item[Coupling] Хамаарал
\item[Decoupling] Салангид байдал
\item[Synchronous communication] Синхрон холбоо
\item[Asynchronous communication] Асинхрон холбоо
\end{description}

\section*{Товчилсон үгс}

\begin{description}
\item[LLM] Large Language Model --- Том хэлний загвар
\item[RAG] Retrieval-Augmented Generation --- Хайлтаар нэмэгдүүлсэн үүсгэлт
\item[API] Application Programming Interface --- Програмчлалын интерфэйс
\item[REST] Representational State Transfer --- Төлөв байдлын төлөөлөлийн дамжуулалт
\item[HTTP] Hypertext Transfer Protocol --- Гипертекст дамжуулалтын протокол
\item[JSON] JavaScript Object Notation --- JavaScript объектын тэмдэглэгээ
\item[SQL] Structured Query Language --- Бүтэцлэгдсэн асуулгын хэл
\item[BERT] Bidirectional Encentations foder Represrom Transformers --- Transformer архитектурт суурилсан хэлний загвар
\item[GPT] Generative Pre-trained Transformer --- Урьдчилан сурсан үүсгэгч Transformer
\item[NLP] Natural Language Processing --- Байгалийн хэлний боловсруулалт
\item[ML] Machine Learning --- Машин сургалт
\item[MLOps] Machine Learning Operations --- Машин сургалтын үйл ажиллагааны удирдлага
\item[EDA] Event-Driven Architecture --- Үйл явдлаар удирдагдах архитектур
\item[SFT] Supervised Fine-tuning --- Удирдлагатай нарийвчилсан сургалт
\item[RLHF] Reinforcement Learning from Human Feedback --- Хүний санал хүсэлтээр бэхжүүлсэн сургалт
\item[DPO] Direct Preference Optimization --- Шууд сонголтын оновчлол
\item[TF-IDF] Term Frequency-Inverse Document Frequency --- Нэр томьёоны давтамж ба баримтын урвуу давтамж
\item[k-NN] k-Nearest Neighbors --- k хамгийн ойр хөршүүд
\item[ANN] Approximate Nearest Neighbors --- Ойролцоо хамгийн ойр хөршүүд
\item[FAISS] Facebook AI Similarity Search --- Facebook-ийн хиймэл оюунт семантик хайлтын сан
\item[gRPC] Google Remote Procedure Call --- Google-ийн алсын процедур дуудлага
\item[SOAP] Simple Object Access Protocol --- Объект хандалтын энгийн протокол
\item[MSE, МХБ] Mongolian Stock Exchange --- Монголын Хөрөнгийн Бирж
\item[CRUD] Create, Read, Update, Delete --- Үүсгэх, унших, шинэчлэх, устгах үйлдлүүд
\item[JWT] JSON Web Token --- JSON форматаар илгээгддэг вэб токен
\item[SSE] Server-Sent Events --- Серверээс илгээгдэх үзэгдлийн урсгал
\item[VaR] Value at Risk --- Эрсдэлийн үнэлгээ
\end{description}

\section*{Техникийн нэр томъёо}

\begin{description}
\item[Docker] Контейнержуулалтын платформ; сервис бүрийг тусдаа орчинд ажиллуулах боломж олгодог.
\item[Apache Kafka] Салангид урсгалын платформ; өндөр дамжуулалттай мессежийн брокер.
\item[Apache Flink] Урсгал боловсруулалтын фреймворк; бодит цагийн өгөгдөл боловсруулах хэрэглүүр.
\item[PostgreSQL] Өгөгдлийн сангийн систем.
\item[Redis] Санах ойд суурилсан өгөгдлийн сан; кэш болон богино хугацааны өгөгдөл хадгалах зориулалттай.
\item[Node.js] JavaScript-ын ажиллах орчин; сервер талын хөгжүүлэлтэд ашиглагдана.
\item[TypeScript] JavaScript-ийн супер багц хэл.
\item[Python] Python Програмчлалын хэл; өгөгдөл боловсруулалт, AI хөгжүүлэлт зэрэгт ашиглагддаг.
\item[Next.js] React суурилсан веб фреймворк.
\item[Express.js] Node.js суурилсан веб фреймворк; RESTful API хөгжүүлэхэд өргөн ашиглагддаг.
\item[Zookeeper] Салангид зохицуулалтын сервис; Kafka зэрэг системийн мета өгөгдөл, кластерийн төлвийг удирдана.
\end{description}

