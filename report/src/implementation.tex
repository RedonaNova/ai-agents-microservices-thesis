\chapter{Хэрэгжүүлэлт}

\section{Удиртгал}

Энэхүү бүлэгт өмнөх бүлгүүдэд судалсан онол зарчим, шийдлийн зохиомжийн дагуу демо системийг бүтээж туршиж үзнэ. Уг демо систем нь монголын хөрөнгийн бирж болон олон улсын хөрөнгийн зах зээлийн мэдээлэлд үндэслэн хиймэл оюун агентуудыг микросервис архитектурт EDA байдлаар нэвтрүүлсэн шинжилгээний веб платформ юм.

Систем нь хэрэглэгчдэд хувьцааны зах зээлийн талаарх мэдээллийг хялбар, ойлгомжтой байдлаар хүргэхийн зэрэгцээ хиймэл оюуны чадавхийг ашиглан хувь хүнд тохирсон дүн шинжилгээ, зөвлөмж өгөх боломжийг олгодог. Уг хэрэгжүүлэлт нь зохиомжийн үр дүнтэй байдлыг шалгаад зогсохгүй судалгааны ажлын хүрээнд бодитоор нэвтрүүлж болох технологийн шийдлийг харуулна.

\section{Зорилго}

Энэхүү хэрэгжүүлэлтийн зорилго нь санал болгож буй шийдлийн ашигтай талыг туршилтаар батлана. Үүнд онолын хэсэгт авч үзсэн хиймэл оюун агентын зохиомж, RAG систем, EDA, Apache Kafka болон Apache Flink зэрэг технологиудыг нэгтгэн уян хатан, өргөтгөх боломжтой, найдвартай платформ бүрдүүлэхийг зорино. Мөн Монголын Хөрөнгийн Биржийн бодит өгөгдлийг ашиглан хэрэглэгчдэд бодит үнэ цэн бүхий шинжилгээ, зөвлөмжийг хүргэх боломжтойг харуулах зэрэг үйлдлүүд хийнэ.

\section{Зорилтууд}

\begin{itemize}
    \item Хиймэл оюун агентуудыг тус тусдаа микросервис хэлбэрээр хөгжүүлж, Docker контейнерт ажиллуулах
    \item Агентуудын хоорондын харилцааг Apache Kafka ашиглан EDA хэлбэрээр хэрэгжүүлж NxM холбоосыг N+M болгон бууруулах
    \item RAG механизмаар агент хоорондын чиглүүлгийг идэвхжүүлэх
    \item Next.js 16 ашиглан орчин үеийн, хариу үйлдэл сайтай веб интерфэйс бүтээх
    \item МХБ болон гадаад хөрөнгийн зах зээлийн бодит өгөгдөл дээр суурилсан хэрэглэгчдэд зориулсан функцүүдийг хэрэгжүүлэх
\end{itemize}

\section{Системийн архитектур}

\begin{figure}[H]
    \centering
    \includegraphics[width=0.9\textwidth]{figures/demoArchitecture.png}
    \caption{Демо системийн архитектур}
    \label{fig:implemented-architecture}
\end{figure}

Хэрэглэгч Nextjs 16 frontend-ээр системд хандана. Frontend нь API Gateway руу HTTP/REST хүсэлт илгээнэ. API Gateway нь үүнийг хүлээн авч, Apache Kafka руу эвент хэлбэрээр дамжуулна. Kafka нь төв мэдээллийн систем болж, бүх агентууд үүгээр дамжин харилцана. Зохион байгуулагч агент нь user.requests сэдвээс хүсэлт уншиж, Gemini AI суурь модел ашиглан хүсэлтийг ангилж, тохирох агент руу чиглүүлнэ. Өөр өөр зорилготой агентууд нь agent.tasks сэдвээс даалгавраа уншиж, боловсруулж, agent.responses сэдэв рүү үр дүнгээ буцаана. API Gateway нь үр дүнг хүлээн авч, SSE эсвэл polling буюу тогтмол агшинд шалгах замаар фронтэнд руу дамжуулна. PostgreSQL 16 нь users, watchlists, mse\_companies, mse\_trading\_history, knowledge\_base зэрэг өгөгдлийг хадгална. Redis 7 нь session болон response caching-д ашиглагдана.

\subsection{Агентуудын үүрэг}

\begin{longtable}{p{0.23\textwidth} p{0.27\textwidth} p{0.42\textwidth}}
\caption{Хэрэгжүүлсэн агентуудын үүрэг} \label{tab:impl-agents-roles} \\
\textbf{Агент} & \textbf{Гол үүрэг} & \textbf{Технологийн чадвар} \\
\hline
\endfirsthead
\caption[]{Хэрэгжүүлсэн агентуудын үүрэг (үргэлжлэл)} \\
\textbf{Агент} & \textbf{Гол үүрэг} & \textbf{Технологийн чадвар} \\
\hline
\endhead
Orchestrator Agent & Зохион байгуулагч & Gemini суурь моделоор хүсэлтүүдийг ангиллаж, агентууд руу даалгавар шилжүүлж, хариу нэгтгэх \\
\hline
Knowledge Agent & Зохион байгуулагчийн санах ой (Бүх агентийн мэдлэг агуулагдана) & RAG систем, vector search, зэргээр хайж дүрмийн дагуу үр дүнгээ нэгтгэнэ  \\
\hline
Investment Agent & Шинжилгээ & МХБ өгөгдөл дүн шинжилж, хөрөнгө оруулалтын зөвлөмж өгөх \\
\hline
News Agent & Мэдээтэй холбоотой үйл ажиллагааг дэмжих & Finnhub API-ээр гадаад мэдээ цуглуулж хураангуйлж утга зүйн анализ хийж дамжуулах \\
\hline
PyFlink Planner & Төлөвлөгч & Үндсэн Kafka consumer/producer-ийг рекурсив эсвэл жирийн чиглүүлэг хийхэд тусална \\
\hline
\end{longtable}

\subsection{Kafka сэдвүүд}

Систем нь 12 Kafka сэдэв ашигладаг. user.requests нь хэрэглэгчийн асуултууд зохион байгуулагч руу илгээгдэх сэдэв юм. user.events нь хэрэглэгчийн бүртгэл, нэвтрэлт, profile өөрчлөлт зэрэг үйл явдлууд бичигдэнэ. agent.tasks нь зохион байгуулагчаас мэргэшсэн агентууд руу даалгавар илгээх сэдэв юм. agent.responses нь агентуудын хариулт буцах сэдэв юм. knowledge.queries нь RAG системд асуулт илгээх сэдэв юм. knowledge.results нь RAG системийн үр дүн буцах сэдэв юм. monitoring.events нь системийн мониторинг, лог бичлэг хийх сэдэв юм. planning.tasks нь PyFlink Planner руу нарийн төвөгтэй даалгавар илгээх сэдэв юм.

\section{Технологийн стек}

Системийн хөгжүүлэлтэд орчин үеийн веб технологиудыг ашиглана.

\subsection{Frontend Технологи}

\begin{longtable}{p{0.25\textwidth} p{0.65\textwidth}}
\caption{Frontend технологийн стек} \label{tab:frontend-tech} \\
\textbf{Технологи} & \textbf{Хэрэглээ} \\
\hline
\endfirsthead
\caption[]{Frontend технологийн стек (үргэлжлэл)} \\
\textbf{Технологи} & \textbf{Хэрэглээ} \\
\hline
\endhead
Nextjs 16 & React-д суурилсан фреймворк, App Router, server-side rendering \\
\hline
React 19 & UI сангийн үндэс, component-based архитектур \\
\hline
TypeScript 5 & Static type checking, compile-time алдаа илрүүлэлт \\
\hline
Tailwind CSS & Utility-first CSS, хурдан responsive дизайн \\
\hline
Shadcn/ui & Radix UI дээр суурилсан accessible компонентууд \\
\hline
\end{longtable}

\subsection{Backend Технологи}

\begin{longtable}{p{0.25\textwidth} p{0.65\textwidth}}
\caption{Backend технологийн стек} \label{tab:backend-tech} \\
\textbf{Технологи} & \textbf{Хэрэглээ} \\
\hline
\endfirsthead
\caption[]{Backend технологийн стек (үргэлжлэл)} \\
\textbf{Технологи} & \textbf{Хэрэглээ} \\
\hline
\endhead
Node.js 20 & API Gateway болон агентуудын runtime орчин \\
\hline
Express.js 4.18 & RESTful API фреймворк, middleware систем \\
\hline
TypeScript 5 & Backend кодын type safety хангах \\
\hline
Python 3.10 & PyFlink Planner агентын хэл \\
\hline
Apache Kafka 3.5 & EDA, message broker \\
\hline
Zookeeper 3.8 & Kafka кластерийн байршуулалт \\
\hline
PostgreSQL 16 & Үндсэн өгөгдлийн сан \\
\hline
Redis 7 & Өгөгдөл хадгалах сав \\
\hline
\end{longtable}

\subsection{Хиймэл оюуны технологи}

\begin{longtable}{p{0.3\textwidth} p{0.6\textwidth}}
\caption{Хиймэл оюуны технологи} \label{tab:ai-tech} \\
\textbf{Технологи} & \textbf{Хэрэглээ} \\
\hline
\endfirsthead
\caption[]{Хиймэл оюуны технологи (үргэлжлэл)} \\
\textbf{Технологи} & \textbf{Хэрэглээ} \\
\hline
\endhead
Google Gemini 2.0, 2.5 Хүсэлтийг ангилах, генерац хийх, хураангуйн боловсруулах \\
\hline
Sentence-Transformers & Text embedding (all-MiniLM-L6-v2 модел, 384-dim vectors) \\
\hline
Finnhub API & Олон улсын зах зээлийн мэдээний эх сурвалж \\
\hline
TradingView Widgets & Хувьцааны тоон үзүүлэлт авах \\
\hline
\end{longtable}

\subsection{Байршуулалт}

Docker болон Docker Compose нь бүх сервисийг контейнержуулахад ашиглагдсан. Агент бүр өөрийн Docker контейнерт ажиллаж, бие даан өргөжих боломжтой. Vercel Hobby Plan ашиглаж демог нэвтрүүлсэн. Хаяг: https://stock-tracker-app-topaz.vercel.app/

\section{Персона тохиолдлууд}

\subsection{Шинэ хэрэглэгч бүртгүүлэх}

\ref{fig:mailRegistration} зургийн дагуу хэрэглэгч бүртгэлийн хуудас руу орж, э-мэйл, нууц үг, нэр (Болд) оруулна.  Дараа нь хөрөнгө оруулалтын зорилго (Өсөлт), эрсдлийн хүлээх чадвар (Дунд), сонирхож буй салбар (Технологи, Санхүү) сонгоно. API Gateway нь POST /api/users/register endpoint-д хүсэлт хүлээн авна. Нууц үг нь bcrypt ашиглан hash хийгдэж, хэрэглэгч PostgreSQL-д хадгалагдана. JWT token үүсгэгдэж, 7 хоногийн хүчинтэй хугацаатай болно. Kafka руу user.events сэдэв-т user.registered event publish хийгдэнэ. Email Service нь уг event-ийг хүлээн авч, хэрэглэгчийн профайл (Өсөлт зорилготой, дунд эрсдэл, технологи ба санхүүгийн салбар сонирхдог) ашиглан Gemini AI-аар хувь хүнд тохирсноор өглөөний мэдээлэл илгээнэ. Имэйлд "Технологийн салбарын хувьцаанд анхаарах нь таны өсөлтийн зорилгод нийцнэ", "Дунд эрсдэлийн профайлд тохирсон портфолио хэрхэн бүтээх вэ" зэрэг зөвлөмжүүд багтана. Имэйл амжилттай илгээгдсэний дараа хэрэглэгч token-тэй системд нэвтрэнэ.

\subsection{Watchlist үүсгэж хувьцаа нэмэх}

\ref{fig:dashboardPage} зургийн дагуу хэрэглэгч нүүр хуудас руу орж, "Миний хувьцаа" нэртэй watchlist харна. POST /api/watchlist хүсэлт JWT токентэй цуг илгээнэ. API Gateway нь хэрэглэгчийг танин баталгаажиж, watchlist-ийг UUID primary key-тэй PostgreSQL-д хадгална. Kafka руу watchlist.created event publish хийгдэнэ. Хэрэглэгч watchlist-даа APU болон BHP нэмэхээр шийднэ. POST /api/watchlist/:id/items хүсэлт хувьцаа бүрийн хувьд илгээгдэнэ. Watchlist items PostgreSQL-д хадгалагдаж, watchlist.item.added event publish хийгдэнэ. Хэрэглэгч watchlist-ээ нээж харахад APU болон BHP хоёр хувьцааг харна. Дараа нь BHP-г хасахаар шийдвэл DELETE /api/watchlist/:id/items/BHP хүсэлт илгээж, watchlist.item.removed event publish хийгдэнэ.

\subsection{Хувьцааны дүн шинжилгээ асуух}

Хэрэглэгч AI чат бот руу орж "APU хувьцааны сүүлийн үнэ, арилжааны хэмжээг харуулаад ирээдүйд өсөх магадлал байгаа юу?" гэж асуулт асуунa. Фронт нь POST /api/agent/query хүсэлт илгээнэ. API Gateway нь unique requestId үүсгэж, user.requests сэдэв руу event publish хийнэ. Frontend нь шууд requestId-тэй хариу авч, SSE холболт үүсгэнэ. Orchestrator Agent нь user.requests сэдвээс event consume хийнэ. Gemini AI ашиглан intent-ийг "market\_analysis" гэж ангилна. Investment Agent руу agent.tasks сэдэв-т даалгавар илгээнэ. Investment Agent нь agent.tasks сэдвээс даалгавраа consume хийнэ. PostgreSQL-ээс APU хувьцааны мэдээллийг query хийнэ: "SELECT symbol, name, last\_price, change\_percent, volume FROM mse\_companies WHERE symbol = 'APU'". Олдсон өгөгдөл (APU, 1280 MNT, +2.4\%, 125340 ширхэг) ашиглан prompt үүсгэнэ. Gemini AI руу prompt илгээж, дүн шинжилгээтэй хариулт авна. Хариулт нь agent.responses сэдэв руу publish хийгдэж, мөн agent\_responses\_cache хүснэгтэд хадгалагдана. API Gateway нь agent.responses сэдвээс хариултыг consume хийж, SSE холболтоор frontend руу stream хийнэ. Frontend нь хариултыг chat interface-д харуулна. Хариулт нь "APU хувьцааны сүүлийн үнэ 1,280 MNT, өмнөх өдрөөс 2.4\% өссөн байна. Арилжааны хэмжээ 125,340 ширхэг нь дундаж түвшинтэй. Уул уурхайн салбарын ерөнхий өсөлтийн чиг хандлагыг харгалзан үзвэл дунд хугацаанд өсөх боломжтой" гэх мэт агуулгатай байна.

\subsection{Өдөр тутмын мэдээ хүлээн авах}

\ref{fig:dailyEmail} зургийн дагуу хэрэглэгч өдөр бүр өөрийн сонирхож буй хувцааний эмэйл мэдэгдэл авах боломжтой. Өглөө 9:00 цагт кроноор POST /api/daily-news/send endpoint дуудна. Систем нь PostgreSQL-ээс бүх идэвхтэй хэрэглэгчдийг авна. Хэрэглэгч бүрийн watchlist-ийн хувьцааны симболуудыг авна. Хэрэглэгчийн watchlist-д APU, TDB, AAPL, MSFT байвал эдгээр хувьцаануудтай холбоотой мэдээг Finnhub API-аас татна. Сүүлийн 24 цагийн 20-30 нийтлэлийг цуглуулж, давхардсан мэдээг хасаж, хамгийн чухал 5-7 нийтлэлийг сонгоно. Gemini AI нь мэдээг хураангуйлж, жагсаалт хэлбэрээр гол агуулгыг гаргана. Утга зүйн шинжилгээ хийж эмэйл дээр эерэг/сөрөг/төвийг сахисан зэрэг харуулна. HTML email template ашиглан мэдээний эхийг хийж өдөр бүр бүх бүртгэгдэлтэй хэрэглэгч рүү эмэйл явуулна.

\section{Хөгжүүлэлтийн явц}

Системийн хэрэгжилт ойролцоогоор 70\% хийгдсэн. Бүрэн хэрэгжсэн хэсгүүд нь Docker, Kafka, PostgreSQL, Redis-ийн суурилуулалт ба агентууд (зохион байгуулагч, хөрөнгө оруулалтын, мэдээний, flink төлөвлөгч), API Gateway, Фронт дээр  https://stock-tracker-app-topaz.vercel.app/ хаягаар байршуулсан байна. 

\subsection{Туршилтын үр дүн}

Гүйцэтгэлийн хувьд датабаазаас мэдээлэл авахад 50-100ms, Kafka мессежийн хугацаа 5-10ms, API Gateway endpoints 200-500ms, суурь модел ашиглан өгөгдөл гаргахад 10-20 секунд, нийт AI ашиглаж дүн шинжилгээний процесс ойролцоогоор 10-17 секунд байна. 

Одоогоор EDA дээрх шинжилгээний API (Kafka → Зохион байгуулагч агент → Хөрөнгө оруулалтын агент → харуй) бүрэн ажиллаж байна. SSE streaming болон polling-аас үр дүнгээ харж байгаа. Monitoring API дээр бүх агент идэвхтэй гэсэн төлөвлөлт харуулж байна.

Өдөр тутмын мэдээний эмэйл илгээх, бүртгэлийн эмэйл илгээх зэрэг хиймэл оюуны тусламжтай агентууд амжилттай ажиллаж байна.
