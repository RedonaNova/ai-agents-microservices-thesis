%!TEX TS-program = xelatex
%!TEX encoding = UTF-8 Unicode
\documentclass[aspectratio=169,10pt]{beamer}

\usepackage{fontspec,xltxtra,xunicode}
\setmainfont[Ligatures=TeX]{Times New Roman}
\setsansfont{Arial}

\usepackage{graphicx}
\usepackage{amsmath,amssymb}
\usepackage{booktabs}
\usepackage{multirow}
\usepackage{multicol}
\usepackage{xcolor}

% Beamer theme
\usetheme{Madrid}
\usecolortheme{default}

% Custom colors
\definecolor{numblue}{RGB}{0,51,102}
\setbeamercolor{structure}{fg=numblue}
\setbeamercolor{title}{fg=numblue}
\setbeamercolor{frametitle}{fg=numblue}

% Remove navigation symbols
\setbeamertemplate{navigation symbols}{}

% Compact spacing
\setlength{\itemsep}{0.2em}
\setlength{\parskip}{0.2em}

% Title information
\title{Микросервис архитектурт суурилсан\\хиймэл оюун агентууд}
\subtitle{Event-Driven Architecture ашиглан уян хатан систем}
\author{Б.Раднаабазар}
\institute{
    Монгол Улсын Их Сургууль\\
    Мэдээллийн Технологи, Электроникийн Сургууль
}
\date{2025 оны 11 сар}
\titlegraphic{\includegraphics[width=1.5cm]{figures/num-logo.png}}

\begin{document}

% Title slide
\begin{frame}
    \titlepage
\end{frame}

% Outline
\begin{frame}{Агуулга}
    \tableofcontents
\end{frame}

% ============================================
% PART 1: INTRODUCTION
% ============================================
\section{Удиртгал}

\begin{frame}{Судалгааны зорилго}
    \begin{block}{Гол зорилго}
        Хиймэл оюун агентуудыг микросервис архитектурт нэвтрүүлэх боломжийг судалж, \textbf{Event-Driven Architecture} ашиглан уян хатан систем бүтээх
    \end{block}
    
    \vspace{0.3cm}
    
    \begin{columns}[t]
        \column{0.5\textwidth}
        \textbf{Гол ойлголтууд:}
        \begin{itemize}
            \item Хиймэл оюуны инженерчлэл
            \item Микросервис архитектур
            \item Event-Driven Architecture
        \end{itemize}
        
        \column{0.5\textwidth}
        \textbf{Технологи:}
        \begin{itemize}
            \item Apache Kafka
            \item Apache Flink
            \item RAG систем
        \end{itemize}
    \end{columns}
\end{frame}

\begin{frame}{Судалгааны зорилтууд}
    \begin{enumerate}
        \item Хиймэл оюуны инженерчлэл, суурь модел, RAG системийг судлах
        \item Хиймэл оюун агентуудын архитектур судлах
        \item Микросервис архитектурын давуу, сул талуудыг тодорхойлох
        \item Агентуудыг микросервис архитектурт нэвтрүүлэхэд тулгарах асуудлуудыг тодорхойлох
        \item Kafka-Flink ашиглан EDA суурилсан архитектур санал болгох
        \item Практик демо систем хөгжүүлж туршиж үзэх
    \end{enumerate}
\end{frame}

% ============================================
% PART 2: THEORETICAL BACKGROUND
% ============================================
\section{Онолын үндэс}

\begin{frame}{Хиймэл оюуны инженерчлэл}
    \begin{columns}[t]
        \column{0.5\textwidth}
        \begin{block}{Уламжлалт ML}
            \begin{itemize}
                \item Модел хөгжүүлэх
                \item Өгөгдөл цуглуулах
                \item Математик мэдлэг
            \end{itemize}
        \end{block}
        
        \column{0.5\textwidth}
        \begin{block}{AI Engineering}
            \begin{itemize}
                \item Бэлэн суурь модел
                \item Программ хангамжид интеграц
                \item Prompt engineering
            \end{itemize}
        \end{block}
    \end{columns}
    
    \vspace{0.2cm}
    \centering
    \includegraphics[width=0.5\textwidth]{figures/aiAndML.png}
\end{frame}

\begin{frame}{Суурь моделийн хөгжил}
    \begin{block}{Хөгжлийн замнал}
        Хэл модел → Том хэлний модел (LLM) → Суурь модел
    \end{block}
    
    \vspace{0.2cm}
    
    \begin{columns}[t]
        \column{0.5\textwidth}
        \textbf{Гол үйл явдлууд:}
        \begin{itemize}
            \item Transformer (2017)
            \item Өөрийгөө удирдсан сургалт
            \item GPT-3: 175 тэрбум параметр
            \item GPT-4: 1.2 их наяд параметр
        \end{itemize}
        
        \column{0.5\textwidth}
        \textbf{Технологи:}
        \begin{itemize}
            \item Attention механизм
            \item Урьдчилан сургалт
            \item Дараах сургалт
            \item Sampling стратеги
        \end{itemize}
    \end{columns}
\end{frame}

\begin{frame}{Хиймэл оюун агентууд}
    \begin{block}{Агент гэж юу вэ?}
        Өөрийн орчныг мэдрэх, түүн дээр үйлдэл хийх чадвартай систем
    \end{block}
    
    \begin{columns}[t]
        \column{0.5\textwidth}
        \centering
        \includegraphics[width=0.85\textwidth]{figures/agentPre.png}
        
        \column{0.5\textwidth}
        \textbf{Гол бүрэлдэхүүн:}
        \begin{itemize}
            \item Орчин
            \item Үйлдэл
            \item Даалгавар
        \end{itemize}
        
        \vspace{0.3cm}
        \textbf{Чадварууд:}
        \begin{itemize}
            \item Төлөвлөлт
            \item Хэрэглүүр ашиглах
            \item RAG систем
        \end{itemize}
    \end{columns}
\end{frame}

\begin{frame}{RAG: Хайлтаар нэмэгдүүлсэн үүсгэлт}
    \centering
    \includegraphics[width=0.65\textwidth]{figures/RAG.png}
    
    \vspace{0.2cm}
    
    \begin{columns}[t]
        \column{0.5\textwidth}
        \begin{block}{Хайгч}
            \begin{itemize}
                \item Нэр томьёо суурилсан
                \item Утга зүй суурилсан
            \end{itemize}
        \end{block}
        
        \column{0.5\textwidth}
        \begin{block}{Үүсгэгч}
            \begin{itemize}
                \item Баримтуудыг ашиглах
                \item Найдвартай мэдээлэл
            \end{itemize}
        \end{block}
    \end{columns}
\end{frame}

% ============================================
% PART 3: MICROSERVICES ARCHITECTURE
% ============================================
\section{Микросервис архитектур}

\begin{frame}{Монолитоос микросервис рүү}
    \begin{columns}[t]
        \column{0.5\textwidth}
        \begin{block}{Монолит}
            \begin{itemize}
                \item Нэг том систем
                \item Нэг кодын сан
                \item Өргөжүүлэх хүндрэлтэй
            \end{itemize}
        \end{block}
        
        \column{0.5\textwidth}
        \begin{block}{Микросервис}
            \begin{itemize}
                \item Жижиг, бие даасан
                \item Тусдаа хөгжүүлэлт
                \item Бие даан өргөжүүлэх
            \end{itemize}
        \end{block}
    \end{columns}
    
    \vspace{0.3cm}
    \begin{alertblock}{Сорилт}
        Сервис хоорондын харилцаа, өгөгдлийн нэгтгэл, динамик routing
    \end{alertblock}
\end{frame}

\begin{frame}{Микросервис хоорондын харилцаа}
    \begin{columns}[t]
        \column{0.5\textwidth}
        \begin{block}{Синхрон}
            \centering
            \includegraphics[width=0.8\textwidth]{figures/microservice.png}
            
            \small
            \begin{itemize}
                \item HTTP REST / gRPC
                \item NxM нягт холбоос
            \end{itemize}
        \end{block}
        
        \column{0.5\textwidth}
        \begin{block}{Асинхрон}
            \centering
            \includegraphics[width=0.8\textwidth]{figures/microserviceEda.png}
            
            \small
            \begin{itemize}
                \item Мессежийн брокер
                \item Салангид байдал
            \end{itemize}
        \end{block}
    \end{columns}
\end{frame}

\begin{frame}{Event-Driven Architecture}
    \centering
    \includegraphics[width=0.65\textwidth]{figures/conMicroservice.png}
    
    \vspace{0.2cm}
    
    \begin{block}{EDA-ын давуу талууд}
        Салангид байдал • Өргөжүүлэх чадвар • Уян хатан байдал • Бодит цагийн боловсруулалт
    \end{block}
\end{frame}

\begin{frame}{Apache Kafka}
    \begin{block}{Үндсэн ойлголтууд}
        \begin{description}[Topic]
            \item[Topic] Үйл явдлуудыг ангилах суваг
            \item[Producer] Үйл явдал үүсгэгч
            \item[Consumer] Үйл явдал хүлээн авагч
            \item[Partition] Өргөжүүлэх, параллель боловсруулалт
        \end{description}
    \end{block}
    
    \vspace{0.2cm}
    
    \begin{block}{Давуу талууд}
        Хэвтээ өргөжих • Бага хоцрогдол • Үйл явдлын хадгалалт • Дахин тоглуулах
    \end{block}
\end{frame}

\begin{frame}{Apache Flink}
    \begin{block}{Flink-ийн давуу талууд}
        \begin{itemize}
            \item \textbf{Төлөв байдлын удирдлага} - Найдвартай тооцоолол
            \item \textbf{Бодит цагийн боловсруулалт} - Үйл явдлын цаг дээр суурилсан
            \item \textbf{Өндөр дамжуулалт} - Секундэд сая сая үйл явдал
            \item \textbf{"Яг нэг" семантик} - Мэдээлэл яг нэг удаа боловсруулагдана
        \end{itemize}
    \end{block}
    
    \vspace{0.2cm}
    
    \begin{block}{Flink ба хиймэл оюун}
        Flink нь LLM-тэй холбогдож, төлөвлөгч агентыг Flink app болгон хөгжүүлэх боломжийг олгодог.
    \end{block}
\end{frame}

% ============================================
% PART 4: PROBLEM AND SOLUTION
% ============================================
\section{Асуудал ба шийдэл}

\begin{frame}{Монолит агентуудын эрсдэл}
    \centering
    
    \vspace{0.2cm}
    
    \begin{alertblock}{Гол асуудлууд}
        \begin{itemize}
            \item \textbf{NxM нягт холбоос} - Өргөжүүлэх хүндрэлтэй
            \item \textbf{Дамжин унах алдаа} - Нэг агент унавал бүх систем унана
            \item \textbf{Хөгжүүлэлтийн удаашрал} - Бүх системийг дахин суурилуулах
        \end{itemize}
    \end{alertblock}
\end{frame}

\begin{frame}{Event-Driven агентын систем}
    \centering
    
    \vspace{0.2cm}
    
    \begin{block}{Гол бүрэлдэхүүн хэсгүүд}
        \begin{enumerate}
            \item \textbf{Эвент үүсгэгчүүд} - Өөр өөр зориулттай агентууд
            \item \textbf{Эвент брокер} - Apache Kafka
            \item \textbf{Эвент хүлээн авагч} - Агентууд, хэрэглүүрүүд
        \end{enumerate}
    \end{block}
\end{frame}

\begin{frame}{Бодит хэрэгжүүлсэн архитектур}
    \centering
    
    \vspace{0.15cm}
    
    \begin{block}{Ажиллах зарчим}
        Хэрэглэгч → API Gateway → Kafka → Orchestrator → Агентууд → SSE → Frontend
    \end{block}
\end{frame}

\begin{frame}{Архитектурын давуу талууд}
    \begin{block}{Техникийн давуу талууд}
        \begin{itemize}
            \item \textbf{Салангид байдал} - Агентууд бие биенээсээ хараат бус
            \item \textbf{Асинхрон харилцаа} - Параллель боловсруулалт
            \item \textbf{Бие даан өргөжүүлэх} - Агент бүр өөрийн хэрэгцээний дагуу
            \item \textbf{Найдвартай байдал} - Үйл явдлууд хадгалагдана
        \end{itemize}
    \end{block}
    
    \vspace{0.2cm}
    
    \begin{block}{Бизнесийн давуу талууд}
        Бодит цагийн шийдвэр гаргалт • Лог хадгалалт • Хэвтээгээр өргөжүүлэх • Шинэ агент нэмэхэд хялбар
    \end{block}
\end{frame}

% ============================================
% PART 5: IMPLEMENTATION
% ============================================
\section{Хэрэгжүүлэлт}

\begin{frame}{Демо системийн архитектур}
    \centering
    \includegraphics[width=0.85\textwidth]{figures/demoArchitecture.png}
    
    \vspace{0.15cm}
    
    \begin{block}{Системийн бүрэлдэхүүн}
        Next.js 16 • API Gateway • Apache Kafka • PostgreSQL 16 • Redis 7
    \end{block}
\end{frame}

\begin{frame}{Бүрэн хэрэгжүүлсэн хэсгүүд}
    \begin{block}{Infrastructure (100\%)}
        \footnotesize
        \begin{itemize}
            \item Docker Compose setup for all services
            \item Apache Kafka 3.5 with Zookeeper
            \item PostgreSQL 16 with pgvector extension
            \item Redis 7 for caching and sessions
            \item 12 Kafka topics created and configured
        \end{itemize}
    \end{block}
    
    \begin{block}{API Gateway (100\%)}
        \footnotesize
        \begin{itemize}
            \item Authentication (register, login, JWT)
            \item Watchlist management (CRUD operations)
            \item AI Agent interaction (query, SSE streaming)
            \item News \& notifications (daily digest)
            \item Monitoring endpoints
        \end{itemize}
    \end{block}
\end{frame}

\begin{frame}{Бүрэн хэрэгжүүлсэн агентууд}
    \begin{columns}[t]
        \column{0.5\textwidth}
        \begin{block}{Orchestrator Agent (100\%)}
            \footnotesize
            \begin{itemize}
                \item Intent classification (6 categories)
                \item Gemini AI integration
                \item Dynamic routing
                \item Request caching
                \item Monitoring events
            \end{itemize}
        \end{block}
        
        \begin{block}{Knowledge Agent (100\%)}
            \footnotesize
            \begin{itemize}
                \item RAG system with vector search
                \item Sentence-Transformers embeddings
                \item PostgreSQL pgvector
                \item Semantic similarity search
            \end{itemize}
        \end{block}
        
        \column{0.5\textwidth}
        \begin{block}{Investment Agent (100\%)}
            \footnotesize
            \begin{itemize}
                \item MSE data integration
                \item Real-time stock analysis
                \item Gemini AI-powered insights
                \item Response caching
            \end{itemize}
        \end{block}
        
        \begin{block}{News Agent (100\%)}
            \footnotesize
            \begin{itemize}
                \item Finnhub API integration
                \item Watchlist-based filtering
                \item Gemini AI summarization
                \item Daily news digest emails
            \end{itemize}
        \end{block}
    \end{columns}
\end{frame}

\begin{frame}{Kafka Topics ба Database}
    \begin{columns}[t]
        \column{0.5\textwidth}
        \begin{block}{Kafka Topics (12)}
            \footnotesize
            \begin{itemize}
                \item \texttt{user.requests}
                \item \texttt{user.events}
                \item \texttt{agent.tasks}
                \item \texttt{agent.responses}
                \item \texttt{knowledge.queries}
                \item \texttt{knowledge.results}
                \item \texttt{planning.tasks}
                \item \texttt{monitoring.events}
            \end{itemize}
        \end{block}
        
        \column{0.5\textwidth}
        \begin{block}{Database Tables (10+)}
            \footnotesize
            \begin{itemize}
                \item \texttt{users}
                \item \texttt{user\_portfolio}
                \item \texttt{watchlists}
                \item \texttt{watchlist\_items}
                \item \texttt{knowledge\_base}
                \item \texttt{mse\_companies}
                \item \texttt{mse\_trading\_history}
                \item \texttt{agent\_responses\_cache}
            \end{itemize}
        \end{block}
    \end{columns}
\end{frame}

\begin{frame}{API Gateway Endpoints}
    \begin{columns}[t]
        \column{0.5\textwidth}
        \begin{block}{Authentication}
            \footnotesize
            \begin{itemize}
                \item \texttt{POST /api/users/register}
                \item \texttt{POST /api/users/login}
                \item \texttt{GET /api/users/profile}
                \item \texttt{PUT /api/users/profile}
            \end{itemize}
        \end{block}
        
        \begin{block}{Watchlist}
            \footnotesize
            \begin{itemize}
                \item \texttt{GET /api/watchlist}
                \item \texttt{POST /api/watchlist}
                \item \texttt{POST /api/watchlist/:id/items}
                \item \texttt{DELETE /api/watchlist/:id}
            \end{itemize}
        \end{block}
        
        \column{0.5\textwidth}
        \begin{block}{AI Agents}
            \footnotesize
            \begin{itemize}
                \item \texttt{POST /api/agent/query}
                \item \texttt{GET /api/agent/response/:id}
                \item \texttt{GET /api/agent/stream/:id}
            \end{itemize}
        \end{block}
        
        \begin{block}{News \& Monitoring}
            \footnotesize
            \begin{itemize}
                \item \texttt{POST /api/daily-news/send}
                \item \texttt{GET /api/monitoring/agents}
                \item \texttt{GET /api/monitoring/metrics}
            \end{itemize}
        \end{block}
    \end{columns}
\end{frame}

\begin{frame}{Orchestrator Agent - Дэлгэрэнгүй}
    \begin{block}{Intent Classification (6 категори)}
        \footnotesize
        \begin{itemize}
            \item \texttt{portfolio\_advice} - Хөрөнгө оруулалтын зөвлөмж
            \item \texttt{market\_analysis} - Зах зээлийн шинжилгээ
            \item \texttt{news\_query} - Мэдээ, сэтгэл хөдлөл
            \item \texttt{historical\_analysis} - Түүхэн өгөгдөл
            \item \texttt{risk\_assessment} - Эрсдэлийн үнэлгээ
            \item \texttt{general\_query} - Ерөнхий асуулт
        \end{itemize}
    \end{block}
    
    \begin{block}{Чадварууд}
        \footnotesize
        \begin{itemize}
            \item Complexity detection (simple vs multi-agent)
            \item Dynamic routing to specialized agents
            \item Request caching for performance
            \item Monitoring event publishing
        \end{itemize}
    \end{block}
\end{frame}

\begin{frame}{Knowledge Agent - RAG System}
    \begin{block}{RAG Features}
        \footnotesize
        \begin{itemize}
            \item Semantic search with vector embeddings
            \item Sentence-Transformers (all-MiniLM-L6-v2)
            \item 384-dimension vectors
            \item PostgreSQL pgvector extension
            \item Cosine similarity search
        \end{itemize}
    \end{block}
    
    \begin{block}{Knowledge Base}
        \footnotesize
        \begin{itemize}
            \item MSE company profiles
            \item Agent capabilities information
            \item Financial domain knowledge
            \item Similarity threshold: 0.7
        \end{itemize}
    \end{block}
\end{frame}

\begin{frame}{Investment Agent ба News Agent}
    \begin{columns}[t]
        \column{0.5\textwidth}
        \begin{block}{Investment Agent}
            \footnotesize
            \begin{itemize}
                \item MSE data integration
                \item Real-time stock analysis
                \item Stock price analysis
                \item Volume trends
                \item Sector performance
                \item Portfolio recommendations
                \item Market overview
                \item Response caching
            \end{itemize}
        \end{block}
        
        \column{0.5\textwidth}
        \begin{block}{News Agent}
            \footnotesize
            \begin{itemize}
                \item Finnhub API integration
                \item Watchlist-based filtering
                \item Gemini AI summarization
                \item Sentiment analysis
                \item Daily news digest emails
                \item HTML email templates
                \item Personalized content
            \end{itemize}
        \end{block}
    \end{columns}
\end{frame}

\begin{frame}{PyFlink Planner ба Frontend}
    \begin{columns}[t]
        \column{0.5\textwidth}
        \begin{block}{PyFlink Planner (70\%)}
            \footnotesize
            \begin{itemize}
                \item Kafka consumer/producer loop
                \item Basic task routing
                \item Consumes from \texttt{planning.tasks}
                \item Publishes to \texttt{planning.results}
                \item ⏳ Stateful computation (pending)
                \item ⏳ Complex event processing (pending)
            \end{itemize}
        \end{block}
        
        \column{0.5\textwidth}
        \begin{block}{Frontend (60\%)}
            \footnotesize
            \begin{itemize}
                \item Next.js 14 App Router
                \item User authentication
                \item Dashboard layout
                \item AI Chat interface
                \item Watchlist management
                \item Responsive design
                \item ⏳ Real-time updates (pending)
            \end{itemize}
        \end{block}
    \end{columns}
\end{frame}

\begin{frame}{Технологийн стек}
    \begin{columns}[t]
        \column{0.33\textwidth}
        \begin{block}{Frontend}
            \footnotesize
            \begin{itemize}
                \item Next.js 16
                \item React 19
                \item TypeScript 5
            \end{itemize}
        \end{block}
        
        \column{0.33\textwidth}
        \begin{block}{Backend}
            \footnotesize
            \begin{itemize}
                \item Node.js 20
                \item Express.js
                \item PostgreSQL 16
                \item Redis 7
            \end{itemize}
        \end{block}
        
        \column{0.33\textwidth}
        \begin{block}{AI \& EDA}
            \footnotesize
            \begin{itemize}
                \item Google Gemini 2.0/2.5
                \item Apache Kafka 3.5
                \item Apache Flink
            \end{itemize}
        \end{block}
    \end{columns}
\end{frame}

\begin{frame}{Хэрэглээний тохиолдол}
    \begin{block}{1. Шинэ хэрэглэгч бүртгүүлэх}
        Бүртгэл үүсгэх → Профайл оруулах → Kafka event → Gemini AI э-мэйл илгээх
    \end{block}
    
    \vspace{0.2cm}
    
    \begin{block}{2. Хувьцааны дүн шинжилгээ}
        Frontend → API Gateway → Kafka → Orchestrator → Investment Agent → SSE streaming
    \end{block}
\end{frame}

\begin{frame}{Системийн дэлгэцүүд}
    \begin{columns}[t]
        \column{0.5\textwidth}
        \centering
        \includegraphics[width=0.85\textwidth]{figures/homePage.png}
        
        \vspace{0.1cm}
        \tiny Нэвтрэх хуудас
        
        \column{0.5\textwidth}
        \centering
        \includegraphics[width=0.85\textwidth]{figures/dashboardPage.png}
        
        \vspace{0.1cm}
        \tiny Нүүр хуудас
    \end{columns}
\end{frame}

\begin{frame}{Хэрэгжүүлэлтийн явц}
    \begin{block}{Хөгжүүлэлт}
        \begin{itemize}
            \item \textbf{Хугацаа:} 6 долоо хоног
            \item \textbf{Хэрэгжүүлэлт:} 70\% бүрэн
        \end{itemize}
    \end{block}
    
    \begin{block}{Бүрэн хэрэгжсэн}
        Docker, Kafka, PostgreSQL, Redis • Бүх агентууд • API Gateway • Frontend (Vercel)
    \end{block}
    
    \begin{block}{Гүйцэтгэл}
        \footnotesize
        Database: 50-100ms • Kafka: 5-10ms • API Gateway: 200-500ms • AI дүн шинжилгээ: 10-17 сек
    \end{block}
\end{frame}

% ============================================
% PART 6: CONCLUSION
% ============================================
\section{Дүгнэлт}

\begin{frame}{Судалгааны үр дүн}
    \begin{block}{Онолын хувьд}
        \begin{itemize}
            \item Хиймэл оюуны инженерчлэл нь програм хангамж хөгжүүлэлтийн шинэ салбар
            \item Суурь модел гарч ирэх нь аппликейшн хөгжүүлэлтийн саад бэрхшээлийг эрс багасгасан
            \item RAG систем нь агентуудын мэдлэгийг өргөтгөж, найдвартай хариулт өгөх боломжийг олгодог
        \end{itemize}
    \end{block}
    
    \begin{block}{Практикийн хувьд}
        \begin{itemize}
            \item Монолит болон API суурилсан холболт нь NxM нягт хамаарал үүсгэдэг
            \item EDA ашиглан агентуудыг салангид микросервис болгон хөгжүүлэх шаардлагатай
        \end{itemize}
    \end{block}
\end{frame}

\begin{frame}{Гол хувь нэмэр}
    \begin{block}{Техникийн хувь нэмэр}
        \begin{itemize}
            \item Хиймэл оюун агентуудыг микросервис архитектурт EDA байдлаар нэвтрүүлэх онол болон практикийн арга зам тодорхойлсон
            \item Бодит систем хэрэгжүүлснээр түүний үр ашигтай байдлыг харуулсан
        \end{itemize}
    \end{block}
    
    \vspace{0.2cm}
    
    \begin{block}{Практик хувь нэмэр}
        МХБ-ийн бодит өгөгдөл дээр суурилсан демо систем • Production орчинд нэвтрүүлж болох технологийн шийдэл
    \end{block}
\end{frame}

\begin{frame}{Хязгаарлалт ба цаашдын судалгаа}
    \begin{block}{Хязгаарлалтууд}
        \begin{itemize}
            \item Демо систем нь үндсэн функцуудыг агуулсан
            \item Портфолио удирдлага, эрсдлийн үнэлгээ зэрэг нарийн төвөгтэй функцууд хараахан хөгжүүлэгдээгүй
        \end{itemize}
    \end{block}
    
    \vspace{0.2cm}
    
    \begin{block}{Цаашдын судалгаа}
        Production орчинд өргөжүүлэх • Өндөр ачааллын тест хийх • Бусад domain дээр хэрэглэх боломжийг судлах
    \end{block}
\end{frame}

\begin{frame}{Дүгнэлт}
    \begin{block}{}
        Энэхүү судалгааны ажил нь хиймэл оюун агентууд болон микросервис архитектурын уялдааг судалж, \textbf{Event-Driven Architecture} ашиглан уян хатан, өргөжих боломжтой систем бүтээх боломжтой гэдгийг онол болон практикийн хувьд харуулсан.
    \end{block}
    
    \vspace{0.3cm}
    
    \begin{block}{}
        Санал болгосон зохиомж нь цаашид хөгжүүлж, production орчинд нэвтрүүлэх суурь болох бөгөөд хиймэл оюуны технологийг микросервис архитектурт нэвтрүүлэх чиглэлд ач холбогдолтой хувь нэмэр болно.
    \end{block}
    
    \vspace{0.3cm}
    
    \centering
    \Large \textbf{Баярлалаа!}
\end{frame}

\begin{frame}{Асуулт хариулт}
    \centering
    \Huge \textbf{Асуулт?}
    
    \vspace{1cm}
    
    \Large Баярлалаа!
\end{frame}

\end{document}
