\begin{abstract}
    
Сүүлийн жилүүдэд хиймэл оюуны салбар дахь технологийн хурдацтай хөгжил нь программ хангамж хөгжүүлэлтийн арга барилд үндсэн өөрчлөлт авчирсан. Суурь загвар гарч ирснээр программ хангамж хөгжүүлэлтийн өмнө тулгардаг саад бэрхшээл эрс буурч, шинэ боломж нээгдэж, хиймэл оюуны инженерчлэл гэсэн шинэ салбар бий болсон ~\cite{huyen2024}. Goldman Sachs-ийн судалгаагаар 2025 он гэхэд АНУ-д хиймэл оюуны хөрөнгө оруулалт 100 тэрбум ам.доллар, дэлхий даяар 200 тэрбум ам.долларт хүрнэ гэж таамаглажээ~\cite{goldman2023}.

Хиймэл оюуны инженерчлэл гэдэг нь бэлтгэгдсэн суурь загвар дээр программ хангамж бүтээх үйл явц юм. Машин сургалт дээр загвар бэлтгэхэд өндөр мэргэжлийн ур чадвар болон асар их өгөгдөл шаардлагатай байсан бол одоо бэлэн загварыг ашиглан аливаа программ хангамжид хиймэл оюуныг интеграци хийх боломжтой болсон нь энэхүү чиг хандлагын гол ач холбогдол юм.
    
Хиймэл оюун агентууд нь хиймэл оюун суурилсан программ хангамжийн хөгжүүлэлтэд онцгой боломжуудыг нээж өгч байна. Агент гэдэг нь өөрийн орчныг мэдрэх, түүн дээр үйлдэл хийх чадвартай систем юм~\cite{huyen2024}. Хиймэл оюун агентууд нь том хэлний загварын хүчийг ашиглан даалгавруудыг ойлгож, төлөвлөгөө гаргаж, олон алхам бүхий үйл ажиллагааг гүйцэтгэх чадвартай. Гэвч байгууллагууд агентуудыг системгүй байршуулах нь мөшгөхөд хүндрэлтэйгээр дамжин унаж, бизнесийн үйл ажиллагаа саатаж байдаг ~\cite{falconer2025future}. Микросервис архитектур нь том системийг жижиг, бие даасан сервисүүдэд тархмал байдлаар хуваах замаар уян хатан, өргөтгөх боломжтой, найдвартай системүүд бий болгодог. Эдгээр агентуудыг микросервис архитектурт нэвтрүүлснээр эдгээр асуудлыг шийдэж, системийн ухаалаг төлөвлөгч, өгөгдөл боловсруулагч, ажлын урсгалын удирдагч зэрэг үүргийг гүйцэтгүүлэх боломжтой ~\cite{huyen2024}.
    
Энэхүү судалгааны ажил нь хиймэл оюун агентууд болон микросервис архитектурын уялдааг судалж, практик шийдлүүдийг санал болгохыг зорьсон. Суурь загварын онол, агентуудын төлөвлөлт, хайлтаар нэмэгдүүлсэн үүсгэлт болон эдгээрийг микросервис архитектурт хэрхэн нэгтгэх талаар авч үзсэн болно.
\end{abstract}
