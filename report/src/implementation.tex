\chapter{Хэрэгжүүлэлт}

\section{Удиртгал}

Энэхүү бүлэгт өмнөх бүлгүүдэд судалсан онолын зарчмуудыг практикт хэрэгжүүлсэн демо системийг дэлгэрэнгүй тайлбарлана. Хэрэгжүүлсэн систем нь Монголын Хөрөнгийн Бирж болон олон улсын хөрөнгийн зах зээлийн мэдээлэлд үндэслэн хиймэл оюун агентуудыг микросервис архитектурт event-driven байдлаар нэвтрүүлсэн санхүүгийн шинжилгээний веб платформ юм.

Систем нь хэрэглэгчдэд хувьцааны зах зээлийн талаарх мэдээллийг хялбар, ойлгомжтой байдлаар хүргэхийн зэрэгцээ хиймэл оюуны чадавхийг ашиглан персоналчилсан дүн шинжилгээ, зөвлөмж өгөх боломжийг олгодог. Уг хэрэгжүүлэлт нь зөвхөн онолын үнэн зөв байдлыг батлаад зогсохгүй bachelor диплом ажлын хүрээнд бодит ашиглагдах чадвартай технологийн шийдлийг харуулна.

\section{Төслийн зорилго}

Демо систем нь дараах гол зорилтуудыг хэрэгжүүлэхээр төлөвлөгдсөн.

\subsection{Үндсэн зорилго}

Үйл явдлаар удирдагдах агентын микросервис архитектурын давуу талыг бодит практикт харуулах. Онолын хэсэгт судалсан хиймэл оюун агент, RAG систем, event-driven архитектур, Apache Kafka болон Apache Flink зэрэг технологиудыг нэгтгэн уян хатан, өргөжих боломжтой, найдвартай систем бүтээх. Монголын Хөрөнгийн Биржийн бодит өгөгдөл ашиглан хэрэглэгчдэд практик үнэ цэнэтэй үйлчилгээ үзүүлэх боломжийг харуулах.

\subsection{Техникийн зорилтууд}

Эхний зорилт нь хиймэл оюун агентуудыг бие даасан микросервис болгон хөгжүүлэх явдал юм. Агент бүр салангид байж, өөрийн тодорхой үүрэгтэй, Docker контейнерт ажиллах ёстой. Хоёр дахь зорилт нь Apache Kafka ашиглан агентуудын хоорондын харилцааг event-driven байдлаар хэрэгжүүлэх. Энэ нь NxM нягт холбоосыг N+M болгон бууруулж, системийн уян хатан байдлыг нэмэгдүүлнэ. Гурав дахь зорилт нь RAG систем ашиглан хиймэл оюунд бодит, үнэн зөв мэдээллээр хангах. Дөрөв дэхь зорилт нь Next.js 14 ашиглан орчин үеийн, responsive веб интерфэйс бүтээх. Тав дахь зорилт нь МХБ-ийн бодит өгөгдөл ашиглан монгол хэрэглэгчдэд тохирсон функцүүдийг хэрэгжүүлэх явдал юм.

\section{Системийн функционал}

Хэрэгжүүлсэн систем нь дараах гол функцуудыг дэмждэг.

\subsection{Хэрэглэгчийн удирдлага}

Хэрэглэгчийн бүртгэл ба нэвтрэлтийн хувьд систем нь имэйл, нууц үг ашиглан бүртгүүлэх боломжийг олгодог. Бүртгүүлэх үед хэрэглэгч хөрөнгө оруулалтын зорилго (өсөлт эсвэл орлого), эрсдлийн хүлээх чадвар (бага, дунд, өндөр), сонирхож буй салбаруудыг (технологи, санхүү, уул уурхай гэх мэт) оруулна. Нууц үг нь bcrypt ашиглан hash хийгдэж, аюулгүй байдлыг хангана. Амжилттай бүртгүүлсний дараа JWT token үүсгэгдэж, 7 хоногийн хүчинтэй хугацаатай болно. Шинээр бүртгүүлсэн хэрэглэгчдэд Google Gemini AI ашиглан түүний профайлд тохирсон персоналчилсан өглөөний мэдээлэл автоматаар илгээгдэнэ. Энэ имэйл нь хэрэглэгчийн сонирхож буй салбар, эрсдлийн хүлээх чадварт нийцсэн зөвлөмж, анхаарах зүйлсийг агуулна.

Хэрэглэгчийн профайл удирдлагын хувьд хэрэглэгч өөрийн хөрөнгө оруулалтын зорилго, эрсдлийн хүлээх чадвар, сонирхож буй салбаруудаа хэдийд ч өөрчлөх боломжтой. Эдгээр өөрчлөлт нь дараагийн AI дүн шинжилгээ, зөвлөмжүүдэд шууд нөлөөлнө.

\subsection{Watchlist систем}

Watchlist удирдлагын хувьд хэрэглэгч олон watchlist үүсгэх боломжтой. Жишээлбэл "Миний уул уурхайн хувьцаа", "Технологийн компаниуд", "Урт хугацааны хөрөнгө оруулалт" гэх мэт нэршлээр watchlist үүсгэж болно. Watchlist бүрт Монголын болон олон улсын хувьцаануудыг нэмэх боломжтой. Монголын хувьцааны жишээ нь APU (Asia Pacific United), TDB (Trade and Development Bank), ERDENET зэрэг байж болно. Олон улсын хувьцааны жишээ нь AAPL (Apple), MSFT (Microsoft), TSLA (Tesla) зэрэг байна. Watchlist-ээс хувьцаа хасах, watchlist устгах зэрэг бүх үндсэн CRUD үйлдлүүд дэмжигдэнэ.

\subsection{Хиймэл оюунтай харилцах}

AI query интерфэйс нь системийн гол онцлог функц юм. Хэрэглэгч байгалийн хэл дээр асуулт асууж болно. Жишээ нь "APU хувьцааны сүүлийн үнэ, арилжааны хэмжээг харуул", "Уул уурхайн салбарын ерөнхий нөхцөл байдлыг тайлбарла", "Технологийн хувьцаанд одоо хөрөнгө оруулах нь зөв үү?" гэх мэт асуултууд асууж болно.

Orchestrator Agent нь хэрэглэгчийн асуултыг Google Gemini AI ашиглан ангилж, тохирох агент руу чиглүүлнэ. Хэрэв асуулт нь хувьцааны дүн шинжилгээтэй холбоотой бол Investment Agent руу, мэдээтэй холбоотой бол News Agent руу, ерөнхий мэдээлэлтэй холбоотой бол Knowledge Agent руу чиглүүлэгдэнэ. Агентууд нь МХБ-ийн бодит өгөгдөл (сүүлийн үнэ, арилжааны хэмжээ, өөрчлөлтийн хувь гэх мэт) ашиглан контекстэд нийцсэн, үнэн зөв хариулт үүсгэнэ. Хариулт нь Server-Sent Events ашиглан бодит цагт stream хэлбэрээр frontend руу дамждаг.

\subsection{Мэдээний үйлчилгээ}

News систем нь хоёр гол функцтэй. Нэгт, хэрэглэгчийн watchlist-д байгаа хувьцаануудтай холбоотой мэдээг Finnhub API-аас татаж авдаг. Finnhub нь олон улсын санхүүгийн мэдээний платформ бөгөөд компани, салбар, зах зээлийн талаарх шинэ мэдээллийг real-time байдлаар өгдөг. Хоёрт, өглөө бүр автоматаар хэрэглэгч бүрт персоналчилсан мэдээний digest илгээгдэнэ. Энэ digest нь хэрэглэгчийн watchlist-тэй холбоотой хамгийн чухал 5-7 мэдээг Gemini AI ашиглан хураангуйлж, sentiment analysis (эерэг, сөрөг, төвийг сахисан) хийж, ойлгомжтой тайлбар нэмсэн байдаг. HTML email template ашигласан байдаг учир зураг, форматтай, уншихад тохиромжтой байдаг.

\subsection{Мэдлэгийн сан}

Knowledge Agent нь RAG (Retrieval-Augmented Generation) систем ашиглан хэрэглэгчдэд санхүүгийн мэргэжлийн мэдлэг өгдөг. МХБ-ийн компаниудын дэлгэрэнгүй мэдээлэл, арилжааны цаг, салбарын онцлог, хөрөнгө оруулалтын зарчим зэрэг мэдээлэл vector embedding хэлбэрээр хадгалагдсан байдаг. Sentence-Transformers модел ашиглан текстийг 384 хэмжээст vector болгон хөрвүүлдэг. PostgreSQL-ийн pgvector extension ашиглан cosine similarity хайлт хийж, асуулттай хамгийн холбоотой мэдээллийг олж авдаг. Энэ мэдлэг нь Investment Agent-ийн хариултыг баяжуулж, илүү нарийвчлалтай, үнэн зөв болгодог.

\section{Хэрэглэгчийн шаардлага}

Системийн хөгжүүлэлтэд дараах хэрэглэгчийн шаардлагууд анхаарагдсан.

\subsection{Функциональ шаардлага}

Хэрэглэгчид аюулгүй нэвтрэх систем хэрэгтэй. JWT token ашиглан session удирдлага, bcrypt ашиглан нууц үгийн аюулгүй байдал хангагдана. Хэрэглэгчид өөрийн watchlist-ийг хялбар удирдах боломж хэрэгтэй. Watchlist үүсгэх, хувьцаа нэмэх, хасах, watchlist устгах үйлдлүүд интуитив байх ёстой. Хэрэглэгчид хувьцааны талаар энгийн хэл дээр асуулт асуух боломж хэрэгтэй. Систем нь асуултыг ойлгож, бодит өгөгдөлд үндэслэн хариулах ёстой. Хэрэглэгчид өөрсдийн сонирхож буй хувьцааны талаарх мэдээг өглөө бүр имэйлээр хүлээн авах хэрэгтэй. Мэдээ нь watchlist-д тохирсон, хураангуйлагдсан, ойлгомжтой байх ёстой.

\subsection{Техникийн шаардлага}

Хариу өгөх хугацааны хувьд энгийн API хүсэлт 500ms-ээс бага, хиймэл оюуны боловсруулалт 15-20 секунд байх ёстой. Системийн найдвартай байдлын хувьд нэг агент унасан ч бусад агентууд хэвийн ажиллах ёстой. Kafka message persistence нь мэдээлэл алдагдахгүйг баталгаажуулна. Өргөжих чадварын хувьд систем нь Kafka partition mechanism ашиглан горизонтал өргөжих боломжтой байх ёстой. Агент нэмэх нь хялбар, систем дахин тохируулалт шаарддаггүй байх ёстой. Аюулгүй байдлын хувьд бүх нууц үг hash хийгдэх, JWT token HMAC-SHA256 ашиглан подпислогдох, бүх API endpoint CORS тохируулагдсан байх ёстой.

\section{Системийн архитектур}

Хэрэгжүүлсэн системийн архитектур нь event-driven microservices pattern дагаж зохион байгуулагдсан.

\subsection{Ерөнхий архитектур}

\begin{figure}[H]
    \centering
    \includegraphics[width=0.9\textwidth]{figures/actual-architecture-placeholder.png}
    \caption{Хэрэгжүүлсэн системийн архитектур}
    \label{fig:implemented-architecture}
\end{figure}

Хэрэглэгч Next.js 14 frontend-ээр системд хандана. Frontend нь API Gateway (Express.js, Node.js 20) руу HTTP/REST хүсэлт илгээнэ. API Gateway нь үүнийг хүлээн авч, Apache Kafka (3.5) руу event хэлбэрээр дамжуулна. Kafka нь төв мэдээллийн систем болж, бүх агентууд энэ дамжуулан харилцана. Orchestrator Agent нь user.requests topic-оос хүсэлт уншиж, Gemini AI ашиглан intent ангилж, тохирох агент руу чиглүүлнэ. Мэргэшсэн агентууд (Investment, Knowledge, News, PyFlink Planner) нь agent.tasks topic-оос даалгавраа уншиж, боловсруулж, agent.responses topic руу үр дүнгээ буцаана. API Gateway нь үр дүнг хүлээн авч, SSE (Server-Sent Events) эсвэл polling замаар frontend руу дамжуулна. PostgreSQL 16 нь users, watchlists, mse\_companies, mse\_trading\_history, knowledge\_base зэрэг өгөгдлийг хадгална. Redis 7 нь session болон response caching-д ашиглагдана.

\subsection{Агентуудын үүрэг}

\begin{longtable}{p{0.23\textwidth} p{0.27\textwidth} p{0.42\textwidth}}
\caption{Хэрэгжүүлсэн агентуудын үүрэг} \label{tab:impl-agents-roles} \\
\textbf{Агент} & \textbf{Гол үүрэг} & \textbf{Технологийн чадвар} \\
\hline
\endfirsthead
\caption[]{Хэрэгжүүлсэн агентуудын үүрэг (үргэлжлэл)} \\
\textbf{Агент} & \textbf{Гол үүрэг} & \textbf{Технологийн чадвар} \\
\hline
\endhead
Orchestrator Agent & Зохион байгуулагч & Gemini AI-аар intent ангилах, агентуудыг идэвхжүүлэх, хариу нэгтгэх \\
\hline
Knowledge Agent & Мэдлэгийн санах ой & RAG систем, vector search, мэдээлэл баяжуулах \\
\hline
Investment Agent & Шинжилгээ & МХБ өгөгдөл дүн шинжилж, хөрөнгө оруулалтын зөвлөмж өгөх \\
\hline
News Agent & Мэдээний шүүлтүүр & Finnhub API, мэдээ цуглуулж хураангуйлах, sentiment analysis \\
\hline
PyFlink Planner & Төлөвлөгч & Үндсэн Kafka consumer/producer loop, энгийн routing \\
\hline
\end{longtable}

\subsection{Kafka Topics}

Систем нь 12 Kafka topic ашигладаг. user.requests нь хэрэглэгчийн асуултууд Orchestrator руу илгээгдэх topic юм. user.events нь хэрэглэгчийн бүртгэл, нэвтрэлт, profile өөрчлөлт зэрэг үйл явдлууд бичигдэнэ. agent.tasks нь Orchestrator-оос мэргэшсэн агентууд руу даалгавар илгээх topic юм. agent.responses нь агентуудын хариулт буцах topic юм. knowledge.queries нь RAG системд асуулт илгээх topic юм. knowledge.results нь RAG системийн үр дүн буцах topic юм. monitoring.events нь системийн мониторинг, лог бичлэг хийх topic юм. planning.tasks нь PyFlink Planner руу нарийн төвөгтэй даалгавар илгээх topic юм.

\subsection{Өгөгдлийн сангийн схем}

PostgreSQL database нь 10 гаруй хүснэгттэй. users хүснэгт нь хэрэглэгчийн мэдээлэл (email, password hash, investment profile) хадгална. watchlists хүснэгт нь хэрэглэгчийн үүсгэсэн watchlist-уудыг UUID primary key-тэй хадгална. watchlist\_items хүснэгт нь watchlist дахь хувьцааны символуудыг хадгална. mse\_companies хүснэгт нь МХБ-ийн компаниудын мэдээлэл (symbol, нэр, салбар, индүстри) хадгална. mse\_trading\_history хүснэгт нь өдөр бүрийн арилжааны түүх (нээлтийн үнэ, хаалтын үнэ, дээд/доод үнэ, хэмжээ) хадгална. knowledge\_base хүснэгт нь RAG системд ашиглагдах мэдлэг, vector embedding-тэй хамт хадгалагдана. agent\_responses\_cache хүснэгт нь агентуудын хариултыг түр хадгалж, дахин ашиглах боломжийг олгоно. monitoring\_events хүснэгт нь системийн лог, metric бичигдэнэ.

\section{Технологийн стек}

Системийн хөгжүүлэлтэд орчин үеийн, production-ready технологиуд ашиглагдсан.

\subsection{Frontend Технологи}

\begin{longtable}{p{0.25\textwidth} p{0.65\textwidth}}
\caption{Frontend технологийн стек} \label{tab:frontend-tech} \\
\textbf{Технологи} & \textbf{Хэрэглээ} \\
\hline
\endfirsthead
\caption[]{Frontend технологийн стек (үргэлжлэл)} \\
\textbf{Технологи} & \textbf{Хэрэглээ} \\
\hline
\endhead
Next.js 14 & React-д суурилсан фреймворк, App Router, server-side rendering \\
\hline
React 18 & UI сангийн үндэс, component-based архитектур \\
\hline
TypeScript 5 & Static type checking, compile-time алдаа илрүүлэлт \\
\hline
Tailwind CSS & Utility-first CSS, хурдан responsive дизайн \\
\hline
Shadcn/ui & Radix UI дээр суурилсан accessible компонентууд \\
\hline
\end{longtable}

\subsection{Backend Технологи}

\begin{longtable}{p{0.25\textwidth} p{0.65\textwidth}}
\caption{Backend технологийн стек} \label{tab:backend-tech} \\
\textbf{Технологи} & \textbf{Хэрэглээ} \\
\hline
\endfirsthead
\caption[]{Backend технологийн стек (үргэлжлэл)} \\
\textbf{Технологи} & \textbf{Хэрэглээ} \\
\hline
\endhead
Node.js 20 & API Gateway болон агентуудын runtime орчин \\
\hline
Express.js 4.18 & RESTful API фреймворк, middleware систем \\
\hline
TypeScript 5 & Backend кодын type safety хангах \\
\hline
Python 3.10 & PyFlink Planner агентын хэл \\
\hline
Apache Kafka 3.5 & Event-driven архитектурын үндэс, message broker \\
\hline
Zookeeper 3.8 & Kafka кластерийн coordination \\
\hline
PostgreSQL 16 & Үндсэн өгөгдлийн сан, pgvector extension-тэй \\
\hline
Redis 7 & Session storage, response caching \\
\hline
\end{longtable}

\subsection{Хиймэл оюуны технологи}

\begin{longtable}{p{0.3\textwidth} p{0.6\textwidth}}
\caption{Хиймэл оюуны технологи} \label{tab:ai-tech} \\
\textbf{Технологи} & \textbf{Хэрэглээ} \\
\hline
\endfirsthead
\caption[]{Хиймэл оюуны технологи (үргэлжлэл)} \\
\textbf{Технологи} & \textbf{Хэрэглээ} \\
\hline
\endhead
Google Gemini 2.0 Flash & Intent classification, response generation, summarization \\
\hline
Sentence-Transformers & Text embedding (all-MiniLM-L6-v2 модел, 384-dim vectors) \\
\hline
pgvector & PostgreSQL-д vector хайлт, cosine similarity \\
\hline
Finnhub API & Олон улсын санхүүгийн мэдээний эх сурвалж \\
\hline
\end{longtable}

\subsection{Deployment Технологи}

Docker 24 болон Docker Compose 2.20 нь бүх сервисийг контейнержуулахад ашиглагдсан. Энэ нь хөгжүүлэлтийн орчин ба production орчны consistency хангадаг. Агент бүр өөрийн Docker контейнерт ажиллаж, бие даан өргөжих боломжтой. Ирээдүйд Kubernetes ашиглан production орчинд deploy хийх бэлтгэл хийгдсэн. GitHub Actions ашиглан CI/CD pipeline үүсгэх боломжтой.

\section{Хэрэглээний тохиолдол}

Системийн бодит ашиглалтыг дараах хэрэглээний тохиолдлуудаар харуулья.

\subsection{Шинэ хэрэглэгч бүртгүүлэх}

Хэрэглэгч registration хуудас руу орж, имэйл (demo@example.com), нууц үг, нэр (Болд) оруулна. Дараа нь хөрөнгө оруулалтын зорилго (Өсөлт), эрсдлийн хүлээх чадвар (Дунд), сонирхож буй салбар (Технологи, Санхүү) сонгоно. API Gateway нь POST /api/users/register endpoint-д хүсэлт хүлээн авна. Нууц үг нь bcrypt ашиглан hash хийгдэж, хэрэглэгч PostgreSQL-д хадгалагдана. JWT token үүсгэгдэж, 7 хоногийн хүчинтэй хугацаатай болно. Kafka руу user.events topic-т user.registered event publish хийгдэнэ. Email Service нь уг event-ийг хүлээн авч, хэрэглэгчийн профайл (Өсөлт зорилготой, дунд эрсдэл, технологи ба санхүүгийн салбар сонирхдог) ашиглан Gemini AI-аар персоналчилсан өглөөний мэдээлэл үүсгэнэ. Имэйлд "Технологийн салбарын хувьцаанд анхаарах нь таны өсөлтийн зорилгод нийцнэ", "Дунд эрсдэлийн профайлд тохирсон портфолио хэрхэн бүтээх вэ" зэрэг зөвлөмжүүд багтана. Имэйл амжилттай илгээгдсэний дараа хэрэглэгч token-тэй системд нэвтрэнэ.

\subsection{Watchlist үүсгэж хувьцаа нэмэх}

Хэрэглэгч dashboard руу нэвтэрч, "Миний уул уурхайн хувьцаа" нэртэй watchlist үүсгэнэ. POST /api/watchlist хүсэлт JWT token-тэй илгээгдэнэ. API Gateway нь хэрэглэгчийг authenticate хийж, watchlist-ийг UUID primary key-тэй PostgreSQL-д хадгална. Kafka руу watchlist.created event publish хийгдэнэ. Хэрэглэгч watchlist-даа APU (Монголын хувьцаа) болон BHP (олон улсын уул уурхайн хувьцаа) нэмэхээр шийднэ. POST /api/watchlist/:id/items хүсэлт хувьцаа бүрийн хувьд илгээгдэнэ. Watchlist items PostgreSQL-д хадгалагдаж, watchlist.item.added event publish хийгдэнэ. Хэрэглэгч watchlist-ээ нээж харахад APU болон BHP хоёр хувьцааг харна. Хожим нь BHP-г хасахаар шийдвэл DELETE /api/watchlist/:id/items/BHP хүсэлт илгээгдэж, watchlist.item.removed event publish хийгдэнэ.

\subsection{Хувьцааны дүн шинжилгээ асуух}

Хэрэглэгч AI chat interface руу орж "APU хувьцааны сүүлийн үнэ, арилжааны хэмжээг харуулаад ирээдүйд өсөх магадлал байгаа юу?" гэж асуулт асуuna. Frontend нь POST /api/agent/query хүсэлт илгээнэ. API Gateway нь unique requestId үүсгэж, user.requests topic руу event publish хийнэ. Frontend нь шууд requestId-тэй хариу авч, SSE холболт үүсгэнэ. Orchestrator Agent нь user.requests topic-оос event consume хийнэ. Gemini AI ашиглан intent-ийг "market\_analysis" гэж ангилна. Investment Agent руу agent.tasks topic-т даалгавар илгээнэ. Investment Agent нь agent.tasks topic-оос даалгавраа consume хийнэ. PostgreSQL-ээс APU хувьцааны мэдээллийг query хийнэ: "SELECT symbol, name, last\_price, change\_percent, volume FROM mse\_companies WHERE symbol = 'APU'". Олдсон өгөгдөл (APU, 1280 MNT, +2.4\%, 125340 ширхэг) ашиглан prompt үүсгэнэ. Gemini AI руу prompt илгээж, дүн шинжилгээтэй хариулт авна. Хариулт нь agent.responses topic руу publish хийгдэж, мөн agent\_responses\_cache хүснэгтэд хадгалагдана. API Gateway нь agent.responses topic-оос хариултыг consume хийж, SSE холболтоор frontend руу stream хийнэ. Frontend нь хариултыг chat interface-д харуулна. Хариулт нь "APU хувьцааны сүүлийн үнэ 1,280 MNT, өмнөх өдрөөс 2.4\% өссөн байна. Арилжааны хэмжээ 125,340 ширхэг нь дундаж түвшинтэй. Уул уурхайн салбарын ерөнхий өсөлтийн чиг хандлагыг харгалзан үзвэл дунд хугацаанд өсөх боломжтой" гэх мэт агуулгатай байна.

\subsection{Өглөөний мэдээний digest хүлээн авах}

Өглөө 9:00 цагт cron job POST /api/daily-news/send endpoint дуудагдана. Систем нь PostgreSQL-ээс бүх идэвхтэй хэрэглэгчдийг query хийнэ. Хэрэглэгч бүрийн watchlist-ийн хувьцааны символуудыг авна. Хэрэглэгчийн watchlist-д APU, TDB, AAPL, MSFT байвал эдгээр хувьцаануудтай холбоотой мэдээг Finnhub API-аас татна. Сүүлийн 24 цагийн 20-30 нийтлэлийг цуглуулж, давхардсан мэдээг хасаж, хамгийн чухал 5-7 нийтлэлийг сонгоно. Gemini AI нь мэдээг хураангуйлж, bullet point хэлбэрээр гол сэдвүүдийг тодруулна. Sentiment analysis хийж (эерэг/сөрөг/төвийг сахисан) илэрхийлнэ. HTML email template ашиглан мэдээний digest бүтээнэ. Имэйлд зах зээлийн ерөнхий байдал, хэрэглэгчийн watchlist-ийн хувьцаануудын гол мэдээ, анхааруулга, сонирхолтой дэлгэрэнгүй мэдээллийн холбоос зэрэг багтана. Email Service хэрэглэгчид имэйл илгээнэ. Хэрэглэгч өглөө сэрэхдээ өөрийн watchlist-тэй холбоотой хамгийн чухал мэдээллийг хүлээн авсан байна.

\section{Хөгжүүлэлтийн явц}

\subsection{Хугацаа ба шат}

Системийн хөгжүүлэлт 2024 оны 11 дүгээр сараас 2025 оны 1 дүгээр сар хүртэл 6 долоо хоногт явагдсан. Эхний 2 долоо хоногт infrastructure (Docker, Kafka, PostgreSQL, Redis) тохируулагдаж, database схем зохиогдож, Kafka topics үүсгэгдсэн. Дараагийн 2 долоо хоногт агентуудын үндсэн функцууд (Orchestrator-ийн intent classification, Knowledge-ийн RAG систем, Investment-ийн МХБ өгөгдөл интеграц, News-ийн Finnhub холболт) хөгжүүлэгдсэн. Дараа нь 1 долоо хоногт API Gateway-ийн бүх endpoints (users, watchlist, agent query, monitoring), JWT authentication, SSE streaming хэрэгжүүлэгдсэн. Эцсийн долоо хоногт Frontend Next.js 14 дээр хөгжүүлэгдэж (authentication UI, dashboard, chat interface, watchlist management), МХБ-ийн seed өгөгдөл нэмэгдэж, эцсийн тестүүд хийгдсэн.

\subsection{Одоогийн байдал}

Системийн хэрэгжилт ойролцоогоор 70\% хийгдсэн байна. Бүрэн хэрэгжсэн хэсгүүд нь Infrastructure (Docker, Kafka, PostgreSQL, Redis бүх тохируулагдсан), Агентууд (Orchestrator, Knowledge, Investment, News, PyFlink Planner үндсэн функцүүдтэй), API Gateway (бүх endpoints ажилладаг), Frontend (authentication, dashboard, chat interface, watchlist management бэлэн), Database (МХБ-ийн seed өгөгдөлтэй) зэргийг агуулна. Хэсэгчлэн хэрэгжсэн хэсгүүд нь PyFlink Planner-ийн нарийн төвөгтэй функцууд (stateful computation, windowing), Frontend-ийн МХБ market overview (layout бэлэн, real-time updates дутмаг), Stock detail pages (үндсэн бүтэц, charts дутмаг), User settings (үндсэн функц, email preferences дутмаг), Monitoring (үндсэн agent status, Prometheus/Grafana дутмаг), Email (welcome ба daily news бэлэн, price alerts дутмаг) зэрэг байна. Хараахан эхлээгүй хэсгүүд нь Portfolio management, Risk assessment (VaR тооцоолол), Advanced analytics (correlation analysis), Real-time market data (WebSocket), Machine learning features (price prediction), Production deployment (Kubernetes, load balancer, auto-scaling) зэрэг ирээдүйн хөгжүүлэлтэд үлдсэн.

\subsection{Туршилтын үр дүн}

Системийн туршилт нь бүх үндсэн функцүүд ажиллаж байгааг баталгаажуулсан. Хэрэглэгчийн бүртгэл нь welcome email-тэй амжилттай. Хэрэглэгчийн нэвтрэлт нь JWT token-тэй ажилладаг. Watchlist CRUD үйлдлүүд бүгд ажилладаг. AI agent query submission болон response хүлээн авалт амжилттай. Event-driven урсгал (Kafka → Orchestrator → Investment Agent → Response) бүрэн ажилладаг. SSE streaming болон polling хоёулаа ажилладаг. Monitoring API нь 5/5 агентыг active гэж зөв харуулж байна. Daily news email илгээлт амжилттай.

Гүйцэтгэлийн хувьд database queries 50-100ms, Kafka message delivery 5-10ms, API Gateway endpoints 200-500ms, LLM inference 10-20 секунд, нийт end-to-end урсгал (AI-тай) ойролцоогоор 17 секунд байна. Эдгээр үр дүн нь bachelor диплом ажлын демод хангалттай гэж үзэж болно.

\section{Бүлгийн дүгнэлт}

Энэхүү бүлэгт санал болгосон event-driven агентын микросервис загварыг бодитоор хэрэгжүүлсэн системийн дэлгэрэнгүй тайлбарыг өгсөн. Хэрэгжүүлэлт нь Монголын Хөрөнгийн Биржийн өгөгдөлд суурилсан санхүүгийн шинжилгээний вэб аппликейшн болж, онол практикийн холбоог тодорхой харуулсан.

Системийн үндсэн зорилго нь event-driven агентын микросервис архитектурын давуу талыг бодит практикт харуулах явдал байсан. Энэ зорилго нь 70\% хэрэгжилттэй амжилттай биелсэн гэж үзэж болно. Хэрэглэгчийн удирдлага, watchlist систем, хиймэл оюунтай харилцах интерфэйс, мэдээний үйлчилгээ, мэдлэгийн сан зэрэг гол функцууд бүрэн ажиллаж байна.

Технологийн стек нь орчин үеийн, production-ready технологиудаас бүрдсэн. Frontend талаас Next.js 14, React 18, TypeScript 5, Tailwind CSS ашиглагдсан. Backend талаас Node.js 20, Express.js, Python 3.10 ашиглагдсан. Infrastructure болгон Apache Kafka 3.5, PostgreSQL 16, Redis 7 сонгогдсон. Хиймэл оюуны технологи болгон Google Gemini 2.0 Flash, Sentence-Transformers ашиглагдсан. Бүх эдгээр технологиуд нь Docker контейнерт ажиллаж, хөгжүүлэлтийн орчин ба production орчны consistency хангадаг.

Системийн архитектур нь онолын бүлгүүдэд судалсан зарчмуудыг шууд дагаж зохион байгуулагдсан. Event-driven architecture нь агентуудын хоорондын NxM нягт холбоосыг N+M болгон бууруулж, салангид байдлыг хангасан. Orchestrator Agent нь төв зохион байгуулагч болж, мэргэшсэн агентууд нь өөр өөрийн үүргээ гүйцэтгэж байна. RAG систем нь Knowledge Agent-д мэргэжлийн мэдлэг өгч, Investment Agent-ийн хариултыг баяжуулж байна. Apache Kafka нь бүх харилцааны үндэс суурь болж, найдвартай, өргөжих боломжтой систем бүтээхэд чухал үүрэг гүйцэтгэж байна.

Хэрэглээний тохиолдлууд нь системийн бодит хэрэглээг харуулсан. Хэрэглэгч бүртгүүлэхээс эхлээд AI дүн шинжилгээ авах хүртэлх бүх урсгал асинхрон, event-driven байдлаар ажиллаж байгаа нь харагдаж байна. Энэ нь зөвхөн онолын хувьд биш, практикт ч event-driven архитектур үр дүнтэй ажилладгийг баталгаажуулж байна.

Хөгжүүлэлтийн явц нь 6 долоо хоногт явагдаж, 70\% хэрэгжилтэд хүрсэн нь bachelor диплом ажлын хүрээнд хангалттай ахиц дэвшил юм. Үлдсэн 30\% нь portfolio management, risk assessment, machine learning features зэрэг нарийн төвөгтэй функцууд бөгөөд эдгээр нь ирээдүйн хөгжүүлэлт эсвэл магистрын түвшний судалгаанд тохиромжтой.

Туршилтын үр дүн нь бүх үндсэн функцууд ажиллаж, event-driven урсгал бүрэн хэрэгжиж, гүйцэтгэл хүлээн зөвшөөрөгдөх түвшинд байгааг харуулсан. Энэ нь санал болгосон загварын үр ашигтай байдлыг практикаар баталгаажуулсан юм.

Эцэст нь энэхүү хэрэгжүүлэлт нь хиймэл оюуны инженерчлэл, агентын архитектур, микросервис, event-driven architecture зэрэг онолын зарчмууд бодит практикт амжилттай хэрэглэгдэж болохыг тодорхой харуулсан. Систем нь зөвхөн bachelor диплом ажлын demo биш, харин цаашид хөгжүүлж, production орчинд нэвтрүүлэх боломжтой суурь болсон гэж дүгнэж болно.
