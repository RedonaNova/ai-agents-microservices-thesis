\chapter{Хэрэгжүүлэлт}

Энэхүү бүлэгт өмнөх бүлгүүдэд судалсан онол, санал болгосон зохиомжийн дагуу демо системийг хөгжүүлж туршина. Эхлээд системийн архитектур, агентуудын үүргийг тайлбарлаад, дараа нь технологийн стек, хэрэгжүүлэлтийн явцыг дэлгэрэнгүй авч үзнэ.

Демо систем нь Монголын Хөрөнгийн Биржийн бодит өгөгдөлд суурилсан хувьцааны шинжилгээний веб платформ юм. Систем нь хиймэл оюун агентуудыг микросервис архитектурт EDA байдлаар нэвтрүүлж, хэрэглэгчдэд хувь хүнд тохирсон дүн шинжилгээ, зөвлөмж өгөх боломжийг олгоно.

\section{Системийн архитектур}

\begin{figure}[H]
    \centering
    \includegraphics[width=0.9\textwidth]{figures/demoArchitecture.png}
    \caption{Демо системийн архитектур}
    \label{fig:implemented-architecture}
\end{figure}

Зураг \ref{fig:implemented-architecture}-т демо системийн архитектурыг дүрсэлсэн. Зураг \ref{fig:implemented-architecture}-аас харахад систем нь дараах бүрэлдэхүүнүүдтэй:

\begin{itemize}
    \item \textbf{Frontend:} Next.js 16 ашиглан хэрэглэгчийн интерфэйс хөгжүүлсэн
    \item \textbf{API Gateway:} HTTP/REST хүсэлтийг хүлээн авч, Kafka руу эвент хэлбэрээр дамжуулна
    \item \textbf{Apache Kafka:} Төв мэдээллийн систем болж, бүх агентууд үүгээр дамжин харилцана
    \item \textbf{Агентууд:} Зохион байгуулагч, хөрөнгө оруулалтын, мэдээний, PyFlink төлөвлөгч агентууд
    \item \textbf{PostgreSQL 16:} Хэрэглэгчид, хувьцааны мэдээлэл, арилжааны түүх хадгална
    \item \textbf{Redis 7:} Session болон response caching-д ашиглагдана
\end{itemize}

\subsection{Агентуудын үүрэг}

\begin{longtable}{p{0.23\textwidth} p{0.27\textwidth} p{0.42\textwidth}}
\caption{Хэрэгжүүлсэн агентуудын үүрэг} \label{tab:impl-agents-roles} \\
\textbf{Агент} & \textbf{Гол үүрэг} & \textbf{Технологийн чадвар} \\
\hline
\endfirsthead
\caption[]{Хэрэгжүүлсэн агентуудын үүрэг (үргэлжлэл)} \\
\textbf{Агент} & \textbf{Гол үүрэг} & \textbf{Технологийн чадвар} \\
\hline
\endhead
Orchestrator Agent & Зохион байгуулагч & Gemini суурь загвараар хүсэлтүүдийг ангиллаж, агентууд руу даалгавар шилжүүлж, хариу нэгтгэх \\
\hline
Knowledge Agent & Зохион байгуулагчийн санах ой (Бүх агентийн мэдлэг агуулагдана) & RAG систем, vector search, зэргээр хайж дүрмийн дагуу үр дүнгээ нэгтгэнэ  \\
\hline
Investment Agent & Шинжилгээ & МХБ өгөгдөл дүн шинжилж, хөрөнгө оруулалтын зөвлөмж өгөх \\
\hline
News Agent & Мэдээтэй холбоотой үйл ажиллагааг дэмжих & Finnhub API-ээр гадаад мэдээ цуглуулж хураангуйлж утга зүйн анализ хийж дамжуулах \\
\hline
PyFlink Planner & Төлөвлөгч & Үндсэн Kafka consumer/producer-ийг рекурсив эсвэл жирийн чиглүүлэг хийхэд тусална \\
\hline
\end{longtable}

\subsection{Kafka сэдвүүд}

Систем нь 12 Kafka сэдэв ашигладаг. user.requests нь хэрэглэгчийн асуултууд зохион байгуулагч руу илгээгдэх сэдэв юм. user.events нь хэрэглэгчийн бүртгэл, нэвтрэлт, profile өөрчлөлт зэрэг үйл явдлууд бичигдэнэ. agent.tasks нь зохион байгуулагчаас мэргэшсэн агентууд руу даалгавар илгээх сэдэв юм. agent.responses нь агентуудын хариулт буцах сэдэв юм. knowledge.queries нь RAG системд асуулт илгээх сэдэв юм. knowledge.results нь RAG системийн үр дүн буцах сэдэв юм. monitoring.events нь системийн мониторинг, лог бичлэг хийх сэдэв юм. planning.tasks нь PyFlink Planner руу нарийн төвөгтэй даалгавар илгээх сэдэв юм.

\section{Технологийн стек}

Системийн хөгжүүлэлтэд орчин үеийн веб технологиудыг ашиглана.

\subsection{Frontend Технологи}

\begin{longtable}{p{0.25\textwidth} p{0.65\textwidth}}
\caption{Frontend технологийн стек} \label{tab:frontend-tech} \\
\textbf{Технологи} & \textbf{Хэрэглээ} \\
\hline
\endfirsthead
\caption[]{Frontend технологийн стек (үргэлжлэл)} \\
\textbf{Технологи} & \textbf{Хэрэглээ} \\
\hline
\endhead
Nextjs 16 & React-д суурилсан фреймворк, App Router, server-side rendering \\
\hline
React 19 & UI сангийн үндэс, component-based архитектур \\
\hline
TypeScript 5 & Static type checking, compile-time алдаа илрүүлэлт \\
\hline
Tailwind CSS & Utility-first CSS, хурдан responsive дизайн \\
\hline
Shadcn/ui & Radix UI дээр суурилсан accessible компонентууд \\
\hline
\end{longtable}

\subsection{Backend Технологи}

\begin{longtable}{p{0.25\textwidth} p{0.65\textwidth}}
\caption{Backend технологийн стек} \label{tab:backend-tech} \\
\textbf{Технологи} & \textbf{Хэрэглээ} \\
\hline
\endfirsthead
\caption[]{Backend технологийн стек (үргэлжлэл)} \\
\textbf{Технологи} & \textbf{Хэрэглээ} \\
\hline
\endhead
Node.js 20 & API Gateway болон агентуудын runtime орчин \\
\hline
Express.js 4.18 & RESTful API фреймворк, middleware систем \\
\hline
TypeScript 5 & Backend кодын type safety хангах \\
\hline
Python 3.10 & PyFlink Planner агентын хэл \\
\hline
Apache Kafka 3.5 & EDA, message broker \\
\hline
Zookeeper 3.8 & Kafka кластерийн байршуулалт \\
\hline
PostgreSQL 16 & Үндсэн өгөгдлийн сан \\
\hline
Redis 7 & Өгөгдөл хадгалах сав \\
\hline
\end{longtable}

\subsection{Хиймэл оюуны технологи}

\begin{longtable}{p{0.3\textwidth} p{0.6\textwidth}}
\caption{Хиймэл оюуны технологи} \label{tab:ai-tech} \\
\textbf{Технологи} & \textbf{Хэрэглээ} \\
\hline
\endfirsthead
\caption[]{Хиймэл оюуны технологи (үргэлжлэл)} \\
\textbf{Технологи} & \textbf{Хэрэглээ} \\
\hline
\endhead
Google Gemini 2.0, 2.5 Хүсэлтийг ангилах, генерац хийх, хураангуйн боловсруулах \\
\hline
Sentence-Transformers & Text embedding (all-MiniLM-L6-v2 загвар, 384-dim vectors) \\
\hline
Finnhub API & Олон улсын зах зээлийн мэдээний эх сурвалж \\
\hline
TradingView Widgets & Хувьцааны тоон үзүүлэлт авах \\
\hline
\end{longtable}

\subsection{Байршуулалт}

Docker болон Docker Compose нь бүх сервисийг контейнержуулахад ашиглагдсан. Агент бүр өөрийн Docker контейнерт ажиллаж, бие даан өргөжих боломжтой. Vercel Hobby Plan ашиглаж демог нэвтрүүлсэн. Хаяг: https://stock-tracker-app-topaz.vercel.app/

\section{Системийн ажиллагааны жишээ}

Энэ хэсэгт системийн гол функцүүд хэрхэн ажилладагийг тайлбарлана.

\subsection{Хувьцааны дүн шинжилгээ}

Хэрэглэгч AI чат бот руу "APU хувьцааны сүүлийн үнэ, арилжааны хэмжээг харуулаад ирээдүйд өсөх магадлал байгаа юу?" гэж асуулт асуухад дараах үйл явц өрнөнө.

Зураг \ref{fig:implemented-architecture}-т үзүүлсэн архитектурын дагуу API Gateway нь хүсэлтийг хүлээн авч, Kafka-ийн user.requests сэдэв рүү илгээнэ. Зохион байгуулагч агент нь Gemini AI ашиглан хүсэлтийг ангилж, хөрөнгө оруулалтын агент руу agent.tasks сэдвээр даалгавар илгээнэ. Хөрөнгө оруулалтын агент нь PostgreSQL-ээс APU хувьцааны мэдээллийг авч, Gemini AI-аар дүн шинжилгээтэй хариулт үүсгэнэ. Хариулт нь agent.responses сэдвээр буцаж, API Gateway SSE холболтоор фронтенд руу дамжуулна.

\subsection{Өдөр тутмын мэдээ илгээх}

Зураг \ref{fig:dailyEmail}-т үзүүлснээр хэрэглэгч өдөр бүр өөрийн сонирхож буй хувьцааны мэдээг э-мэйлээр авах боломжтой. Систем нь хэрэглэгчийн watchlist-ийн хувьцаануудтай холбоотой мэдээг Finnhub API-аас татаж, Gemini AI-аар хураангуйлж, утга зүйн шинжилгээ хийсний дараа э-мэйлээр илгээнэ.

\section{Хөгжүүлэлтийн явц}

Системийн хэрэгжилт ойролцоогоор 70\% хийгдсэн. Бүрэн хэрэгжсэн хэсгүүд нь Docker, Kafka, PostgreSQL, Redis-ийн суурилуулалт ба агентууд (зохион байгуулагч, хөрөнгө оруулалтын, мэдээний, flink төлөвлөгч), API Gateway, Фронт дээр  https://stock-tracker-app-topaz.vercel.app/ хаягаар байршуулсан байна. 

\subsection{Туршилтын үр дүн}

Гүйцэтгэлийн хувьд датабаазаас мэдээлэл авахад 50-100ms, Kafka мессежийн хугацаа 5-10ms, API Gateway endpoints 200-500ms, суурь загвар ашиглан өгөгдөл гаргахад 10-20 секунд, нийт AI ашиглаж дүн шинжилгээний процесс ойролцоогоор 10-17 секунд байна. 

Одоогоор EDA дээрх шинжилгээний API (Kafka → Зохион байгуулагч агент → Хөрөнгө оруулалтын агент → харуй) бүрэн ажиллаж байна. SSE streaming болон polling-аас үр дүнгээ харж байгаа. Monitoring API дээр бүх агент идэвхтэй гэсэн төлөвлөлт харуулж байна.

Өдөр тутмын мэдээний эмэйл илгээх, бүртгэлийн эмэйл илгээх зэрэг хиймэл оюуны тусламжтай агентууд амжилттай ажиллаж байна.
