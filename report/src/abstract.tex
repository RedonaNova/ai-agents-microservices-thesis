\begin{abstract}
Сүүлийн жилүүдэд хиймэл оюун ухааны салбар дахь технологийн хурдацтай хөгжил нь програм хангамж хөгжүүлэлтийн арга барилд үндсэндээ өөрчлөлт авчирсан. Тухайлбал, суурь моделийн (foundation models) гарч ирэх нь аппликейшн хөгжүүлэлтийн өмнө тулгардаг саад бэрхшээлийг эрс багасгасан бөгөөд энэ нь хиймэл оюуны инженерчлэл (AI engineering) гэсэн шинэ салбарыг бий болгоход хүргэжээ. Goldman Sachs-ийн судалгаагаар 2025 он гэхэд АНУ-д Хиймэл оюуны хөрөнгө оруулалт 100 тэрбум ам.доллар, дэлхий даяар 200 тэрбум ам.долларт хүрнэ гэсэн таамаглал дэвшүүлжээ~\cite{goldman2023}.

Хиймэл оюуны инженерчлэл гэдэг нь бэлтгэгдсэн суурь модел дээр аппликейшн бүтээх үйл явц юм. Энэхүү чиг хандлагын ач холбогдол нь хиймэл оюуны аппликейшнуудын эрэлт нэмэгдэхийн зэрэгцээ, тэдгээрийг бүтээх саад бэрхшээл багассан явдал юм. Өмнө нь машин сургалт (machine learning) загвар бэлтгэхэд өндөр мэргэжлийн ур чадвар болон асар их өгөгдлийн иж бүрдэл шаардлагатай байсан бол одоо та бэлэн моделийг ашиглан аппликейшн хөгжүүлж чадна.
    
Энэхүү хөгжил нь микросервис архитектур дээр тулгуурласан програм хангамжийн хөгжүүлэлтэд онцгой боломжуудыг нээж өгч байна. Микросервис архитектур нь том нэг системийг жижиг, бие даасан сервисүүдэд хуваах замаар уян хатан, өргөжүүлэх боломжтой, найдвартай системүүд бий болгодог. Гэвч эдгээр микросервис хоорондын харилцаа холбоо, өгөгдлийн урсгалыг оновчтой удирдах, хэрэглэгчийн хүсэлтийг олон сервисүүдийн хамтын ажиллагаагаар шийдэх нь нарийн төвөгтэй асуудал байсаар ирсэн.
    
Хиймэл оюун агентууд (AI agents) нь энэхүү асуудалд шинэлэг шийдэл санал болж байна. Агент гэдэг нь өөрийн орчныг мэдрэх, түүн дээр үйлдэл хийх чадвартай систем юм~\cite{huyen2024}. Хиймэл оюун агентууд нь том хэлний моделийн (Large Language Models) хүчийг ашиглан даалгавруудыг ойлгож, төлөвлөгөө гаргаж, олон алхам бүхий үйл ажиллагааг гүйцэтгэх чадвартай. Эдгээр агентуудыг микросервис архитектурт нэвтрүүлэх нь системийн ухаалаг орчуулагч, өгөгдөл боловсруулагч, ажлын урсгалын удирдагч зэрэг үүргийг гүйцэтгэх боломжийг олгоно.
    
Энэхүү судалгааны ажил нь хиймэл оюун агентууд болон микросервис архитектурын уялдааг судалж, практик шийдлүүдийг санал болгохыг зорьж байна. Ялангуяа, суурь моделийн онол, агентуудын төлөвлөлт болон үйлдэл хийх механизм, мэдлэг нэмэгдүүлэх арга (Retrieval-Augmented Generation), болон эдгээрийг микросервис архитектурт хэрхэн нэгтгэх талаар авч үзэх юм.    
\end{abstract}