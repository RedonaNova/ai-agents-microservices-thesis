\chapter{Хэрэгжүүлэлт}

Энэхүү бүлэгт өмнөх бүлгүүдэд судалсан онол, санал болгосон зохиомжийн дагуу демо системийг хөгжүүлж туршина. Эхлээд системийн архитектур, агентуудын үүрэг, ажиллах зарчмыг тайлбарлаад, дараа нь хэрэглэгчийн боломжууд, өгөгдлийн урсгал, технологийн стек, туршилтын үр дүнг дэлгэрэнгүй авч үзнэ.

\section{Демо системийн тойм}

Демо систем нь Монголын Хөрөнгийн Биржийн (МХБ) бодит өгөгдөлд суурилсан хувьцааны шинжилгээний веб платформ юм. Систем нь хиймэл оюун агентуудыг микросервис архитектурт үйл явдлаар удирдагдах архитектур (EDA) хэлбэрээр нэвтрүүлж, хэрэглэгчдэд хувь хүнд тохирсон монгол хэл дээрх дүн шинжилгээ, зөвлөмж өгөх боломжийг олгоно.

Системийн гол онцлогууд:
\begin{itemize}
    \item \textbf{Олон агентын архитектур:} Зохион байгуулагч, хөрөнгө оруулалтын, мэдээний, мэдлэгийн, төлөвлөгч гэсэн таван агент
    \item \textbf{Хувийн зөвлөмж:} Хэрэглэгчийн профайл (эрсдлийн хүлээцтэй байдал, хөрөнгө оруулалтын зорилго) дээр суурилсан шинжилгээ
    \item \textbf{Монгол хэлний дэмжлэг:} Бүх хиймэл оюуны хариултууд монгол хэлээр
    \item \textbf{Бодит цагийн өгөгдөл:} МХБ-ийн 2015-2024 оны арилжааны түүх болон одоогийн үнийн мэдээлэл
\end{itemize}

\section{Системийн архитектур}

\begin{figure}[H]
    \centering
    \includegraphics[width=0.9\textwidth]{figures/demoArchitecture.png}
    \caption{Демо системийн архитектур}
    \label{fig:implemented-architecture}
\end{figure}

Зураг \ref{fig:implemented-architecture}-т демо системийн бүрэн архитектурыг дүрсэлсэн. Системийн гол бүрэлдэхүүнүүдийг дараах дарааллаар тайлбарлана.

\subsection{Frontend давхарга}

Next.js 16 фреймворк ашиглан React дээр суурилсан хэрэглэгчийн интерфэйс хөгжүүлсэн. TypeScript хэл ашигласан нь төрлийн аюулгүй байдлыг хангаж, Tailwind CSS болон Shadcn/UI компонентууд ашигласан нь хурдан, уян хатан дизайн хийх боломжийг олгосон. TradingView виджетүүд ашиглан олон улсын хувьцааны график харуулна.

\subsection{API Gateway давхарга}

Express.js 4.18 ашиглан RESTful API бүтээсэн. API Gateway нь хэрэглэгчийн бүх хүсэлтийг хүлээн авч, Kafka руу үйл явдал хэлбэрээр дамжуулах үүрэгтэй. Мөн JWT токен ашиглан хэрэглэгчийн таниулалт хийж, Server-Sent Events (SSE) ашиглан бодит цагийн хариултыг хэрэглэгч рүү дамжуулна.

Гол endpoints:
\begin{itemize}
    \item \texttt{POST /api/agent/query} - Хэрэглэгчийн асуултыг хүлээн авч, зохион байгуулагч руу илгээх
    \item \texttt{GET /api/agent/response/:id} - Агентын хариултыг polling хэлбэрээр авах
    \item \texttt{GET /api/agent/stream/:id} - SSE ашиглан бодит цагийн хариулт авах
    \item \texttt{POST /api/agent/analyze-watchlist} - Ажиглах жагсаалтын МХБ хувьцааг шинжлэх
\end{itemize}

\subsection{Apache Kafka давхарга}

Apache Kafka 3.5 нь системийн төв мэдээллийн сүлжээ болж, бүх агентууд үүгээр дамжин харилцана. Zookeeper 3.8 нь Kafka кластерийн байршуулалтыг удирдана.

\subsection{Агентуудын давхарга}

Таван агент бие даасан микросервис хэлбэрээр Docker контейнерт ажиллана. Агент бүр өөрийн Kafka consumer, producer-тэй бөгөөд бусад агентуудаас хараат бус ажиллах боломжтой.

\subsection{Өгөгдлийн давхарга}

PostgreSQL 16 нь хэрэглэгчид, хувьцааны мэдээлэл, арилжааны түүх, агентуудын хариултыг хадгална. Redis 7 нь session болон response caching-д ашиглагдана.

\section{Агентуудын дэлгэрэнгүй тайлбар}

Энэхүү хэсэгт агент бүрийн үүрэг, ажиллах зарчим, Kafka сэдвүүдтэй харилцах байдлыг дэлгэрэнгүй тайлбарлана.

\subsection{Зохион байгуулагч агент (Orchestrator Agent)}

Зохион байгуулагч агент нь системийн төв тархи болж, хэрэглэгчийн хүсэлтийг ангилж, зохих агент руу чиглүүлэх үүрэгтэй. Энэ нь ReAct (Reasoning and Acting) загварын гол бүрэлдэхүүн юм.

\textbf{Гол үүргүүд:}
\begin{enumerate}
    \item \textbf{Intent Classification:} Google Gemini 2.5 Flash ашиглан хэрэглэгчийн асуултыг portfolio, market\_analysis, news, risk\_assessment, historical\_analysis, general\_query гэсэн ангилалд хуваана
    \item \textbf{Complexity Detection:} Асуулт энгийн эсвэл олон агент шаардсан нарийн төвөгтэй эсэхийг тодорхойлно
    \item \textbf{User Profile Fetching:} PostgreSQL-ээс хэрэглэгчийн investment\_goal, risk\_tolerance, preferred\_industries мэдээллийг авч, хувийн зөвлөмж өгөхөд ашиглана
    \item \textbf{RAG Context Request:} Шаардлагатай бол мэдлэгийн агент руу хүсэлт илгээж, нэмэлт контекст авна
    \item \textbf{Routing:} Энгийн асуултыг шууд agent.tasks руу, нарийн төвөгтэй асуултыг planning.tasks руу илгээнэ
\end{enumerate}

\textbf{Ажиллах урсгал (Код \ref{code:orchestrator}):}

Зохион байгуулагч агент нь user.requests topic-оос хүсэлт хүлээн авч, Gemini AI ашиглан intent ангилна. Дараа нь хэрэглэгчийн профайл авч, нарийн төвөгтэй байдлыг шалгаад, тохирох агент руу чиглүүлнэ.

\begin{lstlisting}[language=JavaScript, caption=Orchestrator Agent main flow, label=code:orchestrator]
const intent = await intentClassifier.classify(query);
const userProfile = await getUserProfile(userId);
const complexity = await complexityDetector.detect(query);

if (complexity.level === 'simple') {
  await routeToAgent(requestId, userId, intent, query);
} else {
  await routeToPlanner(requestId, userId, intent, query);
}
\end{lstlisting}

\textbf{Kafka topics:}
\begin{itemize}
    \item \textbf{Subscribe:} user.requests, knowledge.results
    \item \textbf{Publish:} agent.tasks, planning.tasks, knowledge.queries, monitoring.events
\end{itemize}

\subsection{Хөрөнгө оруулалтын агент (Investment Agent)}

Хөрөнгө оруулалтын агент нь МХБ хувьцааны дүн шинжилгээ хийж, хэрэглэгчийн профайлд суурилсан монгол хэл дээрх хувийн зөвлөмж өгөх үүрэгтэй.

\textbf{Гол үүргүүд:}
\begin{enumerate}
    \item \textbf{МХБ өгөгдөл татах:} PostgreSQL-ээс mse\_trading\_status (одоогийн үнэ) болон mse\_trading\_history (түүхэн өгөгдөл) хүснэгтээс мэдээлэл авна
    \item \textbf{Хувийн зөвлөмж үүсгэх:} Хэрэглэгчийн investmentGoal, riskTolerance, preferredIndustries мэдээлэлд суурилан AI промпт бэлдэнэ
    \item \textbf{Монгол хариулт үүсгэх:} Gemini 2.5 Flash ашиглан монгол хэл дээр товч, мэргэжлийн хариулт үүсгэнэ
    \item \textbf{Хариулт хадгалах:} agent\_responses\_cache хүснэгтэд хариултыг хадгалж, дараа нь polling хийхэд ашиглана
\end{enumerate}

\textbf{Дэмжигдэх үйлдлүүд:}
\begin{itemize}
    \item \texttt{analyze\_portfolio} - Багцын шинжилгээ
    \item \texttt{provide\_advice} - Хөрөнгө оруулалтын зөвлөмж
    \item \texttt{analyze\_market} - Зах зээлийн шинжилгээ
    \item \texttt{analyze\_watchlist} - Ажиглах жагсаалтын хувьцааг шинжлэх
\end{itemize}

\textbf{Хувийн зөвлөмжийн логик (Код \ref{code:personalization}):}

Хэрэглэгчийн профайлаас контекст бүрдүүлж, Gemini AI руу промпт илгээнэ. Эрсдлийн хүлээцтэй байдал "Low" бол аюулгүй хувьцаа, "High" бол өндөр өсөлттэй хувьцаа санал болгоно.

\begin{lstlisting}[language=JavaScript, caption=Personalization context building, label=code:personalization]
const personalizationContext = `
User Profile:
- Goal: ${userProfile.investmentGoal}
- Risk: ${userProfile.riskTolerance}
- Industries: ${userProfile.preferredIndustries}
`;

const prompt = `MSE Analyst. MONGOLIAN, BRIEF.
${personalizationContext}
MSE Data: ${mseData}
Provide advice.`;

const result = await model.generateContent(prompt);
\end{lstlisting}

\subsection{Мэдээний агент (News Agent)}

Мэдээний агент нь Finnhub API-аас олон улсын санхүүгийн мэдээ татаж, утга зүйн шинжилгээ хийж, хураангуй бэлдэх үүрэгтэй.

\textbf{Гол үүргүүд:}
\begin{enumerate}
    \item \textbf{Мэдээ татах:} Finnhub API-аас сүүлийн 7 хоногийн мэдээг татна
    \item \textbf{Sentiment analysis:} Gemini AI ашиглан мэдээ бүрийг positive, negative, neutral гэж ангилна
    \item \textbf{Хураангуй үүсгэх:} Олон мэдээг нэгтгэж, зах зээлийн ерөнхий мэдрэмжийн хураангуй бэлдэнэ
\end{enumerate}

\subsection{Мэдлэгийн агент (Knowledge Agent)}

Мэдлэгийн агент нь RAG (Retrieval-Augmented Generation) системийн үүргийг гүйцэтгэж, мэдлэгийн сангаас холбогдолтой мэдээллийг хайж олно.

\textbf{Гол үүргүүд:}
\begin{enumerate}
    \item \textbf{Мэдлэгийн сан ачаалах:} PostgreSQL-ийн knowledge\_base хүснэгтээс компанийн профайл, бизнес дүрэм, агентуудын чадварын мэдээллийг ачаална
    \item \textbf{Semantic search:} Түлхүүр үгийн тохирол болон metadata фильтер ашиглан хамгийн холбогдолтой баримтуудыг олно
    \item \textbf{Контекст дамжуулах:} Олдсон баримтуудыг knowledge.results сэдвээр буцаана
\end{enumerate}

\textbf{Мэдлэгийн сангийн агуулга:}
\begin{itemize}
    \item МХБ компаниудын профайл (APU, TDB гэх мэт)
    \item Арилжааны дүрэм (арилжааны цаг, хязгаарлалт)
    \item Эрсдлийн менежментийн зөвлөмж
    \item Агент бүрийн чадварын тайлбар (зохион байгуулагчид туслах)
\end{itemize}

\subsection{PyFlink төлөвлөгч агент (Flink Planner)}

PyFlink төлөвлөгч нь нарийн төвөгтэй, олон алхамт даалгаврыг төлөвлөж, гүйцэтгэлийн дараалал үүсгэх үүрэгтэй.

\textbf{Гол үүргүүд:}
\begin{enumerate}
    \item \textbf{Гүйцэтгэлийн төлөвлөгөө үүсгэх:} Gemini AI эсвэл дүрмэнд суурилсан логик ашиглан олон алхамт төлөвлөгөө боловсруулна
    \item \textbf{Агентуудад даалгавар хуваарилах:} Төлөвлөгөөний алхам бүрийг тохирох агент руу agent.tasks сэдвээр илгээнэ
    \item \textbf{Параллель гүйцэтгэл:} Хамааралгүй алхмуудыг зэрэг гүйцэтгэх боломжийг олгоно
\end{enumerate}

\section{Kafka сэдвүүд ба өгөгдлийн урсгал}

Систем нь 8 гол Kafka сэдэв ашиглана. Хүснэгт \ref{tab:kafka-topics}-д сэдэв бүрийн үүргийг тайлбарлав.

\begin{longtable}{p{0.25\textwidth} p{0.65\textwidth}}
\caption{Kafka сэдвүүдийн тайлбар} \label{tab:kafka-topics} \\
\textbf{Сэдэв} & \textbf{Үүрэг} \\
\hline
\endfirsthead
user.requests & Хэрэглэгчийн асуултууд API Gateway-ээс зохион байгуулагч руу \\
\hline
agent.tasks & Зохион байгуулагчаас мэргэшсэн агентууд руу даалгавар \\
\hline
agent.responses & Агентуудын хариулт буцаах \\
\hline
knowledge.queries & Мэдлэгийн агент руу RAG хайлтын хүсэлт \\
\hline
knowledge.results & Мэдлэгийн агентын хайлтын үр дүн \\
\hline
planning.tasks & PyFlink Planner руу нарийн төвөгтэй даалгавар \\
\hline
execution.plans & Төлөвлөгчийн үүсгэсэн гүйцэтгэлийн төлөвлөгөө \\
\hline
monitoring.events & Системийн мониторинг, лог бичлэг \\
\hline
\end{longtable}

\textbf{Өгөгдлийн урсгалын жишээ:}

Хэрэглэгч ``APU хувьцааны шинжилгээ хийж өгнө үү'' гэж асуухад дараах урсгал өрнөнө:

\begin{enumerate}
    \item Frontend $\rightarrow$ API Gateway: HTTP POST /api/agent/query
    \item API Gateway $\rightarrow$ Kafka: user.requests сэдэв рүү мессеж бичих
    \item Orchestrator: user.requests-аас уншиж, intent ангилах (portfolio)
    \item Orchestrator: Хэрэглэгчийн профайл PostgreSQL-ээс авах
    \item Orchestrator $\rightarrow$ Kafka: agent.tasks сэдэв рүү даалгавар илгээх
    \item Investment Agent: agent.tasks-аас уншиж, МХБ өгөгдөл татах
    \item Investment Agent: Gemini AI-аар монгол хариулт үүсгэх
    \item Investment Agent $\rightarrow$ Kafka: agent.responses сэдэв рүү хариулт бичих
    \item Investment Agent $\rightarrow$ PostgreSQL: agent\_responses\_cache-д хадгалах
    \item API Gateway: Polling эсвэл SSE-ээр хариултыг Frontend руу дамжуулах
\end{enumerate}

\section{Өгөгдлийн сангийн бүтэц}

PostgreSQL 16 өгөгдлийн санд дараах гол хүснэгтүүд байна:

\begin{longtable}{p{0.28\textwidth} p{0.62\textwidth}}
\caption{Өгөгдлийн сангийн хүснэгтүүд} \label{tab:db-tables} \\
\textbf{Хүснэгт} & \textbf{Агуулга} \\
\hline
\endfirsthead
users & Хэрэглэгчийн мэдээлэл, профайл (investment\_goal, risk\_tolerance, preferred\_industries) \\
\hline
watchlists & Хэрэглэгчийн ажиглах жагсаалтууд \\
\hline
watchlist\_items & Жагсаалт доторх хувьцаанууд (is\_mse флагтай) \\
\hline
mse\_companies & МХБ-д бүртгэлтэй компаниуд (symbol, name, sector) \\
\hline
mse\_trading\_status & Одоогийн арилжааны үнэ, хэмжээ \\
\hline
mse\_trading\_history & 2015-2024 оны арилжааны түүх \\
\hline
knowledge\_base & RAG системийн мэдлэгийн сан \\
\hline
agent\_responses\_cache & Агентуудын хариултын кэш \\
\hline
monitoring\_events & Системийн мониторингийн лог \\
\hline
\end{longtable}

\section{Хэрэглэгчийн боломжууд}

Систем хэрэглэгчдэд дараах боломжуудыг олгоно:

\subsection{Бүртгэл ба нэвтрэлт}

Хэрэглэгч бүртгүүлэхдээ хөрөнгө оруулалтын профайл бөглөнө:
\begin{itemize}
    \item Хөрөнгө оруулалтын зорилго (Growth, Income, Balanced, Conservative)
    \item Эрсдлийн хүлээцтэй байдал (Low, Medium, High)
    \item Сонирхсон салбарууд (Technology, Finance, Mining гэх мэт)
\end{itemize}

Бүртгэлийн дараа хиймэл оюун ашиглан хувийн мэндчилгээний э-мэйл монгол хэлээр илгээгдэнэ.

\subsection{Ажиглах жагсаалт}

Хэрэглэгч дэлхийн болон МХБ хувьцааг ажиглах жагсаалтад нэмж болно. МХБ хувьцааны хувьд хиймэл оюуны шинжилгээ авах боломжтой.

\subsection{AI чатбот}

Хэрэглэгч монгол эсвэл англи хэлээр асуулт асууж болно. Жишээ асуултууд:
\begin{itemize}
    \item ``APU хувьцааны сүүлийн үнэ ямар байна вэ?''
    \item ``Миний ажиглаж буй хувьцаануудыг шинжилж өгнө үү''
    \item ``Зах зээлийн ерөнхий байдал ямар байна?''
\end{itemize}

\subsection{Өдөр тутмын мэдээ}

Хэрэглэгч өдөр бүр өөрийн ажиглаж буй хувьцааны мэдээг э-мэйлээр авах боломжтой. Finnhub API-аас мэдээ татаж, Gemini AI-аар хураангуйлж илгээнэ.

\section{Технологийн стек}

\subsection{Frontend технологи}

\begin{longtable}{p{0.25\textwidth} p{0.65\textwidth}}
\caption{Frontend технологийн стек} \label{tab:frontend-tech} \\
\textbf{Технологи} & \textbf{Хэрэглээ} \\
\hline
\endfirsthead
Next.js 16 & React-д суурилсан фреймворк, App Router, server-side rendering \\
\hline
React 19 & UI сангийн үндэс, component-based архитектур \\
\hline
TypeScript 5 & Static type checking, compile-time алдаа илрүүлэлт \\
\hline
Tailwind CSS & Utility-first CSS, хурдан responsive дизайн \\
\hline
Shadcn/UI & Radix UI дээр суурилсан accessible компонентууд \\
\hline
TradingView & Олон улсын хувьцааны график, виджетүүд \\
\hline
\end{longtable}

\subsection{Backend технологи}

\begin{longtable}{p{0.25\textwidth} p{0.65\textwidth}}
\caption{Backend технологийн стек} \label{tab:backend-tech} \\
\textbf{Технологи} & \textbf{Хэрэглээ} \\
\hline
\endfirsthead
Node.js 20 & API Gateway болон агентуудын runtime орчин \\
\hline
Express.js 4.18 & RESTful API фреймворк, middleware систем \\
\hline
TypeScript 5 & Backend кодын type safety хангах \\
\hline
Python 3.10 & PyFlink Planner агентын хэл \\
\hline
Apache Kafka 3.5 & Үйл явдлаар удирдагдах архитектурын message broker \\
\hline
Zookeeper 3.8 & Kafka кластерийн зохицуулалт \\
\hline
PostgreSQL 16 & Үндсэн өгөгдлийн сан \\
\hline
Redis 7 & Session болон response caching \\
\hline
Docker & Контейнержуулалт, бие даасан байршуулалт \\
\hline
\end{longtable}

\subsection{Хиймэл оюуны технологи}

\begin{longtable}{p{0.3\textwidth} p{0.6\textwidth}}
\caption{Хиймэл оюуны технологи} \label{tab:ai-tech} \\
\textbf{Технологи} & \textbf{Хэрэглээ} \\
\hline
\endfirsthead
Google Gemini 2.5 Flash & Intent classification, хариулт үүсгэх, sentiment analysis \\
\hline
Finnhub API & Олон улсын санхүүгийн мэдээний эх сурвалж \\
\hline
Nodemailer & Э-мэйл илгээх сервис \\
\hline
\end{longtable}

\section{Туршилтын үр дүн}

\subsection{Гүйцэтгэлийн хэмжилт}

Системийн гүйцэтгэлийг хэмжихэд дараах үр дүн гарсан:

\begin{longtable}{p{0.45\textwidth} p{0.25\textwidth}}
\caption{Гүйцэтгэлийн хэмжилт} \label{tab:performance} \\
\textbf{Үйлдэл} & \textbf{Хугацаа} \\
\hline
\endfirsthead
PostgreSQL-ээс МХБ өгөгдөл татах & 50-100ms \\
\hline
Kafka мессеж дамжуулалт & 5-10ms \\
\hline
API Gateway endpoint хариу & 200-500ms \\
\hline
Gemini AI хариулт үүсгэх & 3-8 секунд \\
\hline
Бүрэн AI шинжилгээний процесс & 5-12 секунд \\
\hline
\end{longtable}

\subsection{Амжилттай туршигдсан функцүүд}

Дараах функцүүд бүрэн ажиллаж байна:
\begin{itemize}
    \item Хэрэглэгчийн бүртгэл, нэвтрэлт, JWT таниулалт
    \item Ажиглах жагсаалт үүсгэх, хувьцаа нэмэх/хасах
    \item МХБ хувьцааны AI шинжилгээ (монгол хэлээр)
    \item Хувийн зөвлөмж (хэрэглэгчийн профайлд суурилсан)
    \item Бүртгэлийн э-мэйл (монгол хэлээр, AI-аар хувийн болгосон)
    \item Өдөр тутмын мэдээний э-мэйл
    \item Агентуудын мониторинг API
    \item SSE болон polling-аар бодит цагийн хариулт авах
\end{itemize}

\subsection{Байршуулалт}

Docker болон Docker Compose ашиглан бүх сервисийг контейнержуулсан. Агент бүр өөрийн Docker контейнерт ажиллаж, бие даан өргөжих боломжтой. Frontend-ийг Vercel Hobby Plan ашиглан байршуулсан. 

Хаяг: \url{https://stock-tracker-app-topaz.vercel.app/}

\section{Бүлгийн дүгнэлт}

Энэхүү бүлэгт хиймэл оюун агентуудыг микросервис архитектурт нэвтрүүлсэн демо системийн хэрэгжүүлэлтийг дэлгэрэнгүй тайлбарлав. Зохион байгуулагч агент нь ReAct загварын гол бүрэлдэхүүн болж, хэрэглэгчийн хүсэлтийг ангилж, хэрэглэгчийн профайл авч, зохих агент руу чиглүүлэх үүргийг амжилттай гүйцэтгэж байна.

Хөрөнгө оруулалтын агент нь МХБ өгөгдөлд суурилан хувийн зөвлөмж монгол хэлээр өгч байгаа нь системийн гол онцлог юм. Мэдээний агент нь Finnhub API-аас мэдээ татаж, утга зүйн шинжилгээ хийж байна. Мэдлэгийн агент нь RAG системийн үүргийг гүйцэтгэж, нэмэлт контекст өгч байна.

Apache Kafka ашигласан нь агентуудыг салангид, бие даасан байлгаж, хэвтээ өргөжих боломжийг олгосон. Ийнхүү онолын хэсэгт судалсан микросервис архитектур, үйл явдлаар удирдагдах архитектур, хиймэл оюун агентуудын зарчмуудыг практикт амжилттай хэрэгжүүлсэн.

