\chapter{Шийдэл ба зохиомж}

Энэхүү бүлэгт олон агентын системийг монолит зохиомжоор хөгжүүлэх эрсдэл ба асуудлуудыг тодорхойлно. Дараа нь тархмал байдлаар агентуудыг микросервис болгон хөгжүүлэх зохиомжийн шийдлийг санал болгоно.

\section{Монолит агентуудын эрсдэл}

Хиймэл оюун хөгжүүлэхэд тогтсон стандарттай фреймворк байдаггүй. ~\cite{falconer2025agents}.  Хиймэл оюуны анхны фреймворкууд нь туршилт хийхэд зориулагдсан бөгөөд бизнесийн бодит орчинд зориулагдаагүй байдаг. Хөгжүүлэгчид notebook дээр загвар туршиж, тодорхой хязгаарлагдмал ажлын урсгалыг ажиллуулж байсан. Гэвч бодит орчинд агентууд нь notebook-д амьдрах нь зохимжгүй. Жишээлбэл өгөгдөл нь хуучрах, өөр дэд процессын үйл ажиллагаа өөрчлөгдвөл алдаа гарах, нэг газар унавал алдаа барихад хэцүү зэрэг асуудал тулгарсан ~\cite{temporal}. Тиймээс агентууд бодит орчинд ажиллаж, бизнесийн үйл ажиллагаатай интеграци хийж, найдвартайгаар өргөжих шаардлагатай байдаг.  

\subsection{Нягт холбогдсон монолит агентуудын архитектур}

Монолит агентын систем нь бүх агентууд нэг аппликейшн доторх модуль эсвэл тархмал бүрэлдэхүүн хэсэг болж ажилладаг.  ~\cite{falconer2025agents}. Хэрэглэгч Orchestrator агент руу хүсэлт илгээхэд Orchestrator нь бусад агентуудыг API-аар дуудах ба агентууд хоорондоо NxM нягт холбоостой байдаг. Энэ нь хялбар харагдах ч хэд хэдэн ноцтой асуудал үүсгэдэг.

\begin{figure}[H]
    \centering
    \includegraphics[width=0.7\textwidth]{figures/agentMonolith.png}
    \caption{Монолит агентын архитектур: NxM нягт холбоос ~\cite{falconer2025agents} ~\cite{temporal}}
    \label{fig:monolith-agent}
\end{figure}

Зураг \ref{fig:monolith-agent}-аас харахад монолит агентууд хоорондоо NxM нягт холбоостой байна. Энэ архитектурын дараах сул талууд байна:

\begin{itemize}
    \item Агентууд бие биеэсээ шууд хамааралтай байдаг учир нэг агентын өөрчлөлт нь бусад агентуудад нөлөөлдөг
    \item Бүх агентууд нэгэн зэрэг өргөжих ёстой бөгөөд тусдаа өргөжүүлэх боломжгүй
    \item Нэг агентыг шинэчлэхэд бүх системийг дахин суурилуулах шаардлагатай
    \item Нэг агент унах нь бүх системийг доголдуулах магадлалтай (дамжин унах алдаа)
\end{itemize}

\section{Микросервис агентуудын шийдэл}

Эдгээр асуудлыг шийдэхийн тулд микросервис архитектурт ашигладаг EDA нь хиймэл оюуны найдвартай нэвтрүүлэлтийн аргачлал болдог. NxM нягт холбоосны оронд агентууд эвент брокероор дамжуулан бие биетэйгээ мессежээр харилцдаг. Энэ нь хамааралыг N+M болгон бууруулж, агентуудыг бүрэн тархмал болгодог.

\begin{figure}[H]
    \centering
    \includegraphics[width=0.9\textwidth]{figures/agentMicroservice.png}
    \caption{EDA орчинд агент хоорондын харилцаа ~\cite{falconer2025future}}
    \label{fig:event-driven-agents}
\end{figure}

Зураг \ref{fig:event-driven-agents}-т үйл олон агентын системийн агент хоорондын харилцааг дүрсэлсэн. Зураг \ref{fig:event-driven-agents}-аас харахад систем нь гурван гол бүрэлдэхүүнтэй:

\begin{itemize}
    \item Эвент үүсгэгчүүд: Өөр өөр зориулалттай агентууд (төлөвлөгч, чиглүүлэгч, мэдээ, төлбөр тооцооны гэх мэт) үйл явдал (kafka event) үүсгэж эвент брокер руу илгээдэг
    \item Эвент брокер: Apache Kafka, RabbitMQ зэрэг брокер нь агентуудын хоорондын төв мэдээллийн систем болж үйл явдлыг хадгалдаг
    \item Эвент хүлээн авагч: Агентууд болон бусад системүүд брокерээс үйл явдлыг хүлээн авч боловсруулна.
\end{itemize}

Энэхүү архитектур нь монолит системийн бүх сул талыг шийддэг. Тархмал байдлын хувьд агентууд эвент брокероор харилцах тул бие биеэсээ хараат бус байдаг. Нэг агент унах нь бусдад шууд нөлөөлөхгүй. Хэрэглүүр, системтэй холбогдсон байдал нь хуучраагүй бодит цагийн өгөгдөл авах зэрэг боломж нээдэг. Асинхрон харилцааны хувьд агентууд бие биенийхээ хариуг хүлээхгүй, параллель боловсруулалт хийх боломжтой. Бие даан өргөжүүлэх чадварын хувьд агент бүр өөрийн цар хүрээний хэрэгцээний тохирсон хэмжээгээр өргөжиж болно. Байршуулалтын тархмал байдлын хувьд агент бүрийг бусдад нөлөөлөхгүйгээр тусдаа суурилуулж болно. Найдвартай байдлын хувьд үйл явдлууд хадгалдаг учир агент түр зуур унасан ч мэдээлэл алдагдахгүй, дахин ачаалуулж болно. Иймээс ч бүх үйл явдал лог байдлаар хадгалагдах учир алдаа олж засах, тест хийх, аудит хийж чадна.

\subsection{Микросервист суурилсан хиймэл оюун агентуудын зохиомжийн жишээ}

\begin{figure}[H]
    \centering
    \includegraphics[width=\textwidth]{figures/charMicroservice.png}
    \caption{Микросервис архитектурд суурилсан олон агент системийн зохиомж ~\cite{falconer2025future} ~\cite{huyen2024}}
    \label{fig:actual-architecture}
\end{figure}

Зураг \ref{fig:actual-architecture}-т санал болгож буй архитектурыг дүрсэлсэн. Зураг \ref{fig:actual-architecture}-аас харахад систем нь дараах байдлаар ажиллана:

\begin{enumerate}
    \item Хэрэглэгч фронтенд дээр хүсэлт илгээнэ
    \item API Gateway хүсэлтийг хүлээн авч Kafka сэдэв рүү үйл явдал болгон илгээнэ
    \item Төлөвлөгч агент суурь загвараар хүсэлтийг ангилж, аль агент руу чиглүүлэхийг тодорхойлно
    \item Өөр өөр зориулалттай агентууд (мэдлэгийн, хөрөнгө оруулалтын, мэдээний гэх мэт) өөрсдийн үүргийг гүйцэтгэнэ
    \item Агентууд үр дүнгээ Kafka сэдэв рүү буцааж илгээнэ
    \item API Gateway хариултыг SSE ашиглан бодит цагийн stream хэлбэрээр хэрэглэгч рүү хүргэнэ
\end{enumerate}

Энэхүү архитектур нь дараах давуу талуудтай:

\begin{itemize}
    \item Параллель боловсруулалт: Агент бүр бие даан ажиллаж, мянга мянган хүсэлтийг зэрэгцээ байдлаар боловсруулна
    \item Уян хатан өргөтгөл: Шинэ агент нэмэх нь бусад агентын кодыг өөрчлөхгүй
    \item Динамик шийдвэр гаргалт: Хиймэл оюун дараагийн алхмыг динамикаар төлөвлөж гүйцэтгэнэ.
    \item Лог хадгалалт: Бүх үйл явдал хадгалагдаж, үнэлгээ, дахин сургалтад ашиглаж болно
\end{itemize}

\section{Бүлгийн дүгнэлт}

Энэхүү бүлэгт олон агентын системийг монолит болон микросервис архитектураар нэвтрүүлэхэд тулгарах асуудлуудыг тодорхойлж, EDA ашиглан тархмал, өргөжих боломжтой систем бүтээх шийдлийг санал болголоо.

Монолит агентын гол асуудлууд нь агентын тоогоор NxM нягт холбоос үүссэнээр өргөжүүлэх хүндэрч, цаашлаад дамжин унах алдаа, хөгжүүлэлтийн удаашрал зэргийг үүсгэдэг. Эдгээр асуудлыг шийдэхийн тулд микросервис архитектурт EDA ашигласнаар агентууд бие биеэсээ салангид, асинхрон байдлаар найдвартай байдлаар ажиллах боломжтой болно. Inngest, Temporal зэрэг одоо байгаа системүүдтэй харьцуулахад энэхүү судалгааны санал болгож буй зохиомж нь анхнаасаа хиймэл оюун агентуудыг тархмал микросервис болгон хөгжүүлж, Kafka-Flink ашиглан бодит цагийн урсгал боловсруулалт хийж, нээлттэй эх суурилсан технологи ашиглаж, хэвтээгээр өргөжих зэрэг онцлогтой.

Уг судалгааны ажлын технологийн архитектур нь Apache Kafka-г агентуудын мэдээлэл солилцох суваг болгон ашиглаж тархмал байдлыг хангаж, асинхрон харилцаагаар хүлээлт үүсгэхгүй, олон хэрэглэгч дэмжих, найдвартай байдлыг хангах, алдаатай хүсэлтийг дахин ачааллуулах зэрэг үйлдлүүдийг EDA-ын тусламжтайгаар хэрэгжүүлнэ. Энэ архитектурын зохиомж нь бие даасан ухаалаг микросервис болгон хөгжүүлэх боломжийг олгоно. 