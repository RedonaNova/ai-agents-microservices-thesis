\chapter{Хэрэгжүүлэлт}

Энэхүү бүлэгт санал болгож буй үйл явдлаар удирдагдах агентын микросервис загварыг бодитоор хэрэгжүүлсэн системийн дэлгэрэнгүй тайлбарыг өгнө. Хэрэгжүүлэлт нь Монголын Хөрөнгийн Биржийн өгөгдөл дээр суурилсан санхүүгийн шинжилгээний вэб аппликейшн болох бөгөөд агентын микросервис архитектурын бодит хэрэглээг харуулна.

\section{Төслийн тойм}

Демо систем нь дараах гол зорилтуудыг хэрэгжүүлнэ. Эхний зорилт нь үйл явдлаар удирдагдах архитектур ашиглан хиймэл оюун агентуудыг салангид микросервис болгон хөгжүүлэх явдал юм. Хоёр дахь зорилт нь Apache Kafka, Apache Flink ашиглан бодит цагийн асинхрон харилцаа хэрэгжүүлэх. Гурав дахь зорилт нь МХБ-ийн бодит өгөгдөл ашиглан хөрөнгө оруулалтын зөвлөмж өгөх. Дөрөв дэхь зорилт нь монолит агентаас илүү өргөжих боломжтой, найдвартай системийн давуу талыг харуулах. Эцсийн зорилт нь орчин үеийн програм хангамжийн онол болон практикийн холбоог бодит демогоор харуулах явдал юм.

Системийн гол давуу тал нь дараах байдалтай. Микросервис салангид байдал нь агент бүр бие даан хөгжүүлэгдэж, тест хийгдэж, байршуулагддаг. Үйл явдлаар удирдагдах архитектур нь Kafka ашиглан агентууд асинхрон харилцдаг. Хиймэл оюуны интеграц нь Gemini AI, RAG систем агентуудад ухаалаг шийдвэр гаргах чадвар өгдөг. Бодит өгөгдөл ашиглалт нь МХБ-ийн бодит компани, арилжааны өгөгдөл дээр суурилсан дүн шинжилгээ өгдөг. Өргөжих чадвар нь Kafka partition, consumer group механизм ашиглан горизонтал өргөжүүлэлт хялбар.

\subsection{Хөгжүүлэлтийн хугацаа}

Системийн хөгжүүлэлт 2024 оны 11 дүгээр сараас 2025 оны 1 дүгээр сар хүртэл 6 долоо хоногт үргэлжилсэн. Эхний 2 долоо хоногт infrastructure (Docker, Kafka, PostgreSQL, Redis) тохируулагдаж, database схем зохиогдсон. Дараагийн 2 долоо хоногт агентуудын үндсэн функцүүд (Orchestrator, Knowledge, Investment, News) хөгжүүлэгдсэн. Дараа нь 1 долоо хоногт API Gateway, JWT authentication, SSE streaming хэрэгжүүлэгдсэн. Эцсийн долоо хоногт Frontend Next.js 14 дээр хөгжүүлэгдэж, МХБ-ийн seed өгөгдөл нэмэгдсэн. Одоогийн байдлаар системийн 70\% хэрэгжсэн байна.

\section{Санал болгосон архитектур}

Хэрэгжүүлсэн систем нь агентуудыг тархи мэт мэдрэхүйн системээр зохион байгуулж, үйл явдлаар удирдагдах архитектур ашиглан салангид байлгаж, уян хатан, өргөжих боломжтой, найдвартай систем бүтээнэ.

\subsection{Агентуудын үүрэг}

\begin{longtable}{p{0.23\textwidth} p{0.27\textwidth} p{0.42\textwidth}}
\caption{Хэрэгжүүлсэн агентуудын үүрэг} \label{tab:impl-agents-roles} \\
\textbf{Агент} & \textbf{Гол үүрэг} & \textbf{Технологийн чадвар} \\
\hline
\endfirsthead
\caption[]{Хэрэгжүүлсэн агентуудын үүрэг (үргэлжлэл)} \\
\textbf{Агент} & \textbf{Гол үүрэг} & \textbf{Технологийн чадвар} \\
\hline
\endhead
Orchestrator Agent & Зохион байгуулагч & Gemini AI-аар intent ангилах, агентуудыг идэвхжүүлэх, хариу нэгтгэх \\
\hline
Knowledge Agent & Мэдлэгийн санах ой & RAG систем, vector search, мэдээлэл баяжуулах \\
\hline
Investment Agent & Шинжилгээ & МХБ өгөгдөл дүн шинжилж, хөрөнгө оруулалтын зөвлөмж өгөх \\
\hline
News Agent & Мэдээний шүүлтүүр & Finnhub API, мэдээ цуглуулж хураангуйлах, sentiment analysis \\
\hline
PyFlink Planner & Төлөвлөгч & Нарийн төвөгтэй даалгавар задлах, stateful боловсруулалт \\
\hline
\end{longtable}

\subsection{Архитектурын бүрэлдэхүүн}

\begin{longtable}{p{0.17\textwidth} p{0.23\textwidth} p{0.52\textwidth}}
\caption{Системийн бүрэлдэхүүн хэсгүүд} \label{tab:impl-architecture-components} \\
\textbf{Бүрэлдэхүүн} & \textbf{Технологи} & \textbf{Хэрэгжүүлсэн функц} \\
\hline
\endfirsthead
\caption[]{Системийн бүрэлдэхүүн хэсгүүд (үргэлжлэл)} \\
\textbf{Бүрэлдэхүүн} & \textbf{Технологи} & \textbf{Хэрэгжүүлсэн функц} \\
\hline
\endhead
API Gateway & Node.js, Express & JWT authentication, SSE streaming, REST endpoints \\
\hline
Orchestrator & TypeScript, Gemini AI & Intent classification, dynamic routing, cache механизм \\
\hline
Knowledge & TypeScript, pgvector & Sentence-Transformers embedding, semantic search \\
\hline
Investment & TypeScript, Gemini AI & МХБ өгөгдөл татах, LLM prompt үүсгэх, хариулт кэш хийх \\
\hline
News & TypeScript, Finnhub & News fetch, summarization, daily digest email \\
\hline
PyFlink Planner & Python 3.10, Kafka & Kafka consumer/producer loop, simplified planning \\
\hline
Kafka & Apache Kafka 3.5 & 12 topics, Snappy compression, consumer groups \\
\hline
PostgreSQL & PostgreSQL 16 & 10+ хүснэгт, pgvector extension, ACID transactions \\
\hline
Redis & Redis 7 & Session storage, caching layer \\
\hline
Frontend & Next.js 14, React 18 & SSE client, JWT storage, responsive UI \\
\hline
\end{longtable}

\section{Системийн шаардлагууд}

Демо систем нь дараах үндсэн шаардлагуудыг хангах ёстой.

\subsection{Функциональ шаардлага}

Хэрэглэгчийн удирдлагын талаас авч үзвэл, систем нь хэрэглэгч бүртгүүлэх, нэвтрэх, профайл засварлах үндсэн функцүүдийг дэмжих шаардлагатай. Хэрэглэгчийн нууц үг нь bcrypt ашиглан hash хийгдэх бөгөөд JWT token ашиглан session удирдлага хийгдэнэ. Шинээр бүртгүүлсэн хэрэглэгчдэд Gemini AI-аар бүтээсэн персоналчилсан өглөөний мэдээлэл илгээгдэнэ.

Watchlist удирдлагын хувьд хэрэглэгчид олон watchlist үүсгэх, устгах, нэр өөрчлөх боломжтой байх ёстой. Watchlist бүрт дэлхийн болон МХБ-ийн хувьцаануудыг нэмэх, хасах боломжтой. Системийн хамгийн гол функц болох агентуудтай харилцах хэсэг нь хэрэглэгчид байгалийн хэл дээр асуулт асуух боломж өгөх ёстой. Orchestrator Agent нь асуултын сэдлийг тодорхойлж, тохирох агентууд руу чиглүүлнэ. Агентууд хамтран ажиллаж, контекстэд тохирсон, МХБ-ийн бодит өгөгдөл дээр үндэслэсэн хариулт өгнө. Хариулт нь Server-Sent Events ашиглан бодит цагт stream хэлбэрээр хүрдэг.

Санхүүгийн мэдээний хувьд News Agent нь Finnhub API-тай холбогдож, хэрэглэгчийн watchlist-тэй холбоотой мэдээг цуглуулна. Gemini AI ашиглан мэдээг хураангуйлж, sentiment analysis хийнэ. Өглөө бүр автоматаар персоналчилсан мэдээний digest илгээгдэнэ. МХБ-ийн өгөгдлийн хувьд систем нь МХБ-д бүртгэлтэй компаниудын мэдээлэл (нэр, салбар, тайлбар), хувьцааны өдөр бүрийн арилжааны өгөгдөл (нээлтийн үнэ, хаалтын үнэ, дээд/доод үнэ, хэмжээ), түүхэн өгөгдөл (trend шинжилгээнд ашиглагдана) зэргийг хадгална.

\subsection{Техникийн шаардлага}

Архитектурын хувьд агент бүр бие даасан микросервис болох ёстой. Агентууд Kafka ашиглан асинхрон харилцана. Нэг агентын алдаа бусад агентуудад шууд нөлөөлөхгүй байх ёстой. Агент бүр Docker контейнерт ажиллах бөгөөд бие даан өргөжих боломжтой байна.

Гүйцэтгэлийн хувьд хэрэглэгчийн асуултаас бүрэн хариулт хүртэлх хугацаа 3-5 секундын дотор байх ёстой. Системийн дамжуулалт секундэд 10-15 хүсэлт боловсруулах чадвартай байна. LLM inference нь хамгийн их хугацаа эзэлж, Kafka messaging болон database queries нь хурдан (100-300ms) ажиллана.

Найдвартай байдлын хувьд Kafka message persistence нь мэдээлэл алдагдахгүйг баталгаажуулна. Агент унасан ч дахин сэргээгдэж, алдсан мессежүүдийг боловсруулах боломжтой. Database нь ACID property дагаж, өгөгдлийн бүрэн бүтэн байдлыг хангана. JWT token нь HMAC-SHA256 алгоритм ашиглан подпислогдоно.

Хэмжих боломжийн хувьд бүх агент Kafka-д мониторингийн үйл явдал илгээх ёстой. Prometheus metrics export дэмжигдэнэ. Бүх HTTP API endpoint нь JSON хэлбэрийн алдааны мэдээлэл буцаана. Logging нь Winston (Node.js) болон Python logging ашиглан нэгдсэн форматтай хийгдэнэ.

\section{Технологийн стек}

Хэрэгжүүлсэн системийн технологийн стек нь орчин үеийн, production-ready технологиудаас бүрдэнэ.

\subsection{Frontend Технологи}

Frontend нь Next.js 14 App Router ашиглан хөгжүүлэгдсэн. Энэ нь React 18 дээр суурилсан орчин үеийн фреймворк бөгөөд server-side rendering, file-based routing, API routes зэрэг функцүүдийг өгдөг. TypeScript 5 нь static type checking, improved developer experience, compile-time error detection зэрэг давуу талуудыг өгч байна. Tailwind CSS нь utility-first CSS фреймворк бөгөөд хурдан, responsive дизайн бүтээх боломжийг олгодог. Shadcn/ui нь Radix UI дээр суурилсан accessible, customizable компонент сан болох бөгөөд орчин үеийн интерфэйс бүтээхэд тусалдаг.

\subsection{Backend Технологи}

API Gateway нь Express.js 4.18 дээр суурилсан Node.js 20 ашигладаг. Энэ нь RESTful API endpoints, JWT authentication middleware, CORS тохиргоо, rate limiting зэрэг функцүүдийг хангадаг. Агентуудын хувьд Orchestrator, Knowledge, Investment, News агентууд нь Node.js 20, TypeScript 5 ашигладаг. Тэд Kafkajs сан ашиглан Kafka-тай харилцдаг. PyFlink Planner Agent нь Python 3.10 ашиглан Apache Flink 1.18 дээр суурилсан stream processing хийдэг.

Message Broker болох Apache Kafka 3.5 нь event-driven архитектурын үндэс суурь юм. Zookeeper 3.8 нь Kafka кластерийн coordination хариуцдаг. Өгөгдлийн сангийн хувьд PostgreSQL 16 нь үндсэн өгөгдлийн сан бөгөөд users, watchlists, mse\_companies, mse\_trading\_data, knowledge\_base, agent\_responses\_cache зэрэг хүснэгтүүдийг агуулна. Redis 7 нь кэш болон session удирдлагад ашиглагддаг.

\subsection{Хиймэл оюуны технологи}

Google Gemini 2.0 Flash модел нь Orchestrator, Investment, News агентуудад ашиглагддаг. Энэ модел нь intent classification, response generation, news summarization зэрэг функцүүдэд ашиглагдана. Temperature параметр 0.7 нь баттай боловч бүтээлч хариулт өгөх боломжийг олгоно. Sentence-Transformers (all-MiniLM-L6-v2 модел) нь Knowledge Agent-д текстийг 384 хэмжээст vector embedding болгон хөрвүүлэхэд ашиглагддаг. PostgreSQL-ийн pgvector extension нь cosine similarity хайлт хийхэд ашиглагддаг.

\subsection{Гадаад API}

Finnhub API нь дэлхийн санхүүгийн мэдээ авахад ашиглагддаг. Компани бүрийн мэдээ, ерөнхий зах зээлийн мэдээ, sentiment индикатор зэрэг мэдээлэл авдаг. NewsAPI нь нэмэлт мэдээний эх сурвалж болж чаддаг. Alpha Vantage API нь нөөцөд бэлэн байгаа.

\subsection{Deployment Технологи}

Docker 24 ба Docker Compose 2.20 нь бүх сервисийг контейнержуулахад ашиглагддаг. Энэ нь хөгжүүлэлтийн орчин ба production орчны consistency хангадаг. Үйлдвэрлэлийн орчинд Kubernetes ашиглаж болох бэлэн. GitHub Actions нь CI/CD pipeline-д ашиглагдаж болох бөгөөд автомат тест, build, deploy хийх боломжтой.

% TODO: User will provide technology stack diagram/photo

\section{Хэрэглээний тохиолдол}

Демо систем нь дараах гол хэрэглээний тохиолдлуудыг дэмждэг.

\subsection{Хэрэглэгчийн бүртгэл ба нэвтрэлт}

Шинэ хэрэглэгч системд бүртгүүлэхдээ имэйл, нууц үг, нэр, хөрөнгө оруулалтын зорилго, эрсдлийн хүлээх чадвар, сонирхож буй салбарууд зэрэг мэдээллээ оруулна. Систем нь хэрэглэгчийн өгөгдлийг PostgreSQL-д хадгалж, нууц үгийг bcrypt ашиглан hash хийнэ. JWT token үүсгэж, 7 хоногийн хүчинтэй хугацаатай буцаана. Kafka-д user.registered үйл явдал publish хийнэ. Email Service нь уг үйл явдлыг хүлээн авч, Gemini AI ашиглан хэрэглэгчийн профайлд тохирсон персоналчилсан өглөөний мэдээлэл үүсгэж илгээнэ.

Одоо байгаа хэрэглэгч нэвтрэхдээ имэйл болон нууц үгээ оруулна. Систем нь bcrypt ашиглан нууц үгийг шалгаж, JWT token үүсгэнэ. Сүүлд нэвтэрсэн цагийг шинэчилж, Kafka-д user.login үйл явдал publish хийнэ.

\subsection{Watchlist удирдлага}

Хэрэглэгч "Mining Stocks" нэртэй шинэ watchlist үүсгэнэ. Систем нь watchlist-ийг өөрийн мэдээлэл (owner\_id, name, timestamps) зэрэгтэй хамт үүсгэнэ. UUID primary key ашиглан давхцалгүй таних тэмдэг бүтээнэ. watchlist.created үйл явдлыг Kafka-д publish хийнэ.

Хэрэглэгч APU (Монголын хөрөнгийн бирж), ERDENET хувьцаануудыг watchlist-даа нэмнэ. Watchlist items хүснэгтэд symbol, is\_mse флаг (МХБ эсэхийг тэмдэглэх), added\_at timestamp зэрэг мэдээлэл хадгалагдана. watchlist.item.added үйл явдал нэмэгдсэн хувьцаа бүрийн хувьд publish хийгдэнэ.

Дараа нь хэрэглэгч APU хувьцааг хасахаар шийднэ. Систем нь ownership шалгаж, watchlist items-ээс устгана. watchlist.item.removed үйл явдал publish хийгдэнэ. Эцэст нь хэрэглэгч watchlist-ийг бүхэлд нь устгахаар шийднэ. CASCADE delete механизм нь watchlist дахь бүх items-ыг автоматаар устгана. watchlist.deleted үйл явдал publish хийгдэнэ.

\subsection{Агенттай харилцах}

Хэрэглэгч "APU хувьцааны сүүлийн үнэ, арилжааны хэмжээг харуул" гэж асуулт асуухад frontend нь POST /api/agent/query request илгээж, request body-д query text, user ID, timestamp агуулагдана. API Gateway нь unique request ID үүсгэж, JSON мессеж бүтээж user-requests Kafka topic руу publish хийнэ.

Orchestrator Agent нь user-requests topic-оос мессеж consume хийж, Google Gemini 2.0 Flash ашиглан intent classification хийж "market\_analysis" гэж тодорхойлно. Тохирох агент (Investment Agent) болон шаардлагатай параметрүүдийг тодорхойлж, orchestrator-tasks topic руу даалгавар илгээнэ.

Investment Agent нь orchestrator-tasks topic-оос өөрийн даалгаврыг consume хийж, PostgreSQL-ээс APU хувьцааны мэдээллийг query хийнэ. МХБ-ийн бодит өгөгдөл (сүүлийн үнэ ₮1,280, өөрчлөлт +2.4\%, арилжааны хэмжээ 125,340 ширхэг) агуулсан prompt үүсгэж Google Gemini руу илгээнэ. LLM дүн шинжилгээтэй хариулт үүсгэж, үр дүнг agent-responses topic руу publish хийх ба agent\_responses\_cache хүснэгтэд хадгална.

API Gateway нь agent-responses topic-оос хариултыг consume хийж, SSE холболт байвал шууд frontend руу stream хийнэ. Polling хүсэлт ирвэл cache-аас хариултыг буцаана. Frontend нь хариултыг хүлээн авч chat interface-д харуулна.

\subsection{Өдөр бүрийн мэдээний digest}

Өглөө 9:00 цагт scheduler (cron job эсвэл manual trigger) нь POST /api/daily-news/send endpoint дуудаж, систем нь PostgreSQL-ээс бүх идэвхтэй хэрэглэгчдийн жагсаалтыг авч, хэрэглэгч бүрийн watchlist-ийг query хийнэ.

News Agent нь Finnhub API-аас watchlist-ийн хувьцаануудын (жишээ нь AAPL, MSFT, ERDENET) сүүлийн 24 цагийн мэдээг татна. Олон эх сурвалжаас 20-30 нийтлэл цуглуулж, давхардсан мэдээг хасаж, хамгийн чухал 5-7 нийтлэлийг сонгоно.

Gemini AI нь мэдээг хураангуйлж, bullet point хэлбэрээр гол сэдвүүдийг тодруулна. Sentiment analysis хийж (positive/negative/neutral) илэрхийлнэ. "Bottom Line" тайлбар нэмж, жирийн хөрөнгө оруулагчдад ойлгомжтой болгоно. HTML email template ашиглан мэдээний хураангуй бүтээнэ.

Email Service нь хэрэглэгч бүрт персоналчилсан мэдээний digest илгээнэ. Email дотор 📊 Market Overview, 📈 Top Gainers, 📉 Sector News, 💡 Bottom Line зэрэг хэсгүүд байна. Хэрэглэгч өглөө сэрэхдээ өөрийн watchlist-тэй холбоотой хамгийн чухал мэдээг хүлээн авна.

\section{Хэрэгжүүлэлтийн явц}

Одоогийн байдлаар системийн ойролцоогоор 70\% хэрэгжсэн байна.

\subsection{Бүрэн хэрэгжсэн хэсгүүд}

Infrastructure талаас Docker Compose ашиглан Kafka, Zookeeper, PostgreSQL, Redis зэргийг контейнержуулсан. Kafka topics (user-requests, orchestrator-tasks, agent-responses, monitoring-events гэх мэт) бүх үүсгэгдсэн. PostgreSQL database schema (users, watchlists, watchlist\_items, mse\_companies, mse\_trading\_data, knowledge\_base, agent\_responses\_cache) бүрэн хэрэгжсэн. Database migrations автоматжуулагдсан.

Агентуудын талаас Orchestrator Agent нь Google Gemini 2.0 Flash ашиглан intent classification хийх, agent routing хийх, response aggregation хийх чадвартай. Knowledge Agent нь RAG систем (vector embedding, semantic search) хэрэгжүүлсэн, PostgreSQL-ийн pgvector extension ашигладаг, санхүүгийн баримт бичгүүдийг боловсруулж хадгалдаг. Investment Agent нь МХБ-ийн бодит өгөгдөлд үндэслэн дүн шинжилгээ хийдэг, Gemini AI ашиглан хариулт үүсгэдэг, stock analysis, portfolio recommendations, market overview зэрэг функцтэй. News Agent нь Finnhub API холбогдсон, watchlist-тэй холбоотой мэдээ шүүж авдаг, Gemini AI ашиглан summarization болон sentiment analysis хийдэг, өдөр бүрийн email digest илгээдэг. PyFlink Planner Agent нь үндсэн Kafka consumer/producer loop хэрэгжүүлсэн, demonstration зориулалттай энгийн төлөвлөгч ажилладаг.

API Gateway-н талаас бүх RESTful API endpoints хэрэгжсэн. User registration/login (POST /api/users/register, POST /api/users/login), Watchlist CRUD (POST /api/watchlist, GET /api/watchlist, POST /api/watchlist/:id/items, DELETE /api/watchlist/:id/items/:symbol), Agent query (POST /api/agent/query), Agent response polling (GET /api/agent/response/:requestId), Daily news (POST /api/daily-news/send), Monitoring (GET /api/monitoring/agents, GET /api/monitoring/metrics) зэрэг. JWT authentication middleware хэрэгжсэн. Server-Sent Events (SSE) streaming дэмжигдэнэ. Kafka producer/consumer integration ажилладаг.

Frontend-ийн талаас Next.js 14 App Router бүтэц бүрэн. User authentication (register, login, JWT storage) хэрэгжсэн. Dashboard layout болон navigation ажилладаг. AI Chat interface үндсэн функцтэй (message history, SSE streaming, typing indicators). Watchlist management page бэлэн. Responsive дизайн (mobile, tablet, desktop) хэрэгжсэн.

Database-ийн талаас МХБ-ийн компаниудын мэдээлэл (APU, TDB, ERDENET гэх мэт) хадгалагдсан. Seed data үүсгэгдсэн. Historical trading data хадгалах схем бэлэн.

\subsection{Хэсэгчлэн хэрэгжсэн хэсгүүд}

PyFlink Planner Agent нь одоогоор demonstration зориулалттай энгийн хувилбартай. Stateful computation, windowing, complex event processing зэрэг нарийн төвөгтэй функцүүд хойш үлдсэн. Frontend-ийн хувьд МХБ market overview page үндсэн layout бэлэн боловч бодит цагийн price updates хараахан хэрэгжээгүй. Stock detail pages схем бэлэн боловч дэлгэрэнгүй charts, technical indicators дутагдалтай. User settings page үндсэн функцтэй боловч email preferences, notification settings нэмэх шаардлагатай. Monitoring ба analytics нь үндсэн agent status check ажилладаг боловч Prometheus metrics, Grafana dashboard, detailed performance analytics хараахан байхгүй. Email notifications нь welcome email болон daily news digest ажилладаг боловч price alerts, portfolio rebalancing notifications хараахан байхгүй.

\subsection{Хараахан эхлээгүй хэсгүүд}

Portfolio management функц нь портфолио үүсгэх, хувьцаа худалдан авах/зарах, performance tracking зэрэг функцууд хараахан эхлээгүй. Risk assessment функц нь VaR (Value at Risk) тооцоолох, portfolio diversification шинжилгээ, stress testing scenarios зэрэг функцууд төлөвлөгдсөн боловч хэрэгжээгүй. Advanced analytics нь historical trend analysis, correlation analysis, sector rotation analysis зэрэг функцууд цаашдын хөгжүүлэлтэд үлдсэн. Real-time market data нь WebSocket холболт ашиглан бодит цагийн price updates, live trading volume, market sentiment indicators зэрэг функцууд төлөвлөгдсөн боловч хараахан байхгүй. Machine learning features нь price prediction модел, anomaly detection, personalized recommendations зэрэг функцууд ирээдүйн сайжруулалтад үлдсэн. Production deployment нь Kubernetes orchestration, load balancer, auto-scaling, monitoring stack (Prometheus + Grafana), centralized logging (ELK Stack) зэрэг production-ready deployment хараахан хийгдээгүй.

\section{Кодын жишээ}

Системийн гол бүрэлдэхүүн хэсгүүдийн кодын жишээг үзүүлэх нь техникийн хэрэгжүүлэлтийг илүү ойлгомжтой болгоно.

\subsection{Orchestrator Agent: Intent Classification}

Orchestrator Agent-ийн хамгийн чухал функц нь хэрэглэгчийн хүсэлтийг ангилах явдал юм. Доорх код нь Gemini AI ашиглан intent classification хийх логикийг харуулж байна.

\begin{lstlisting}[language=JavaScript, caption=Intent classification using Gemini AI]
// File: backend/orchestrator-agent/src/intent-classifier.ts

import { GoogleGenerativeAI } from '@google/generative-ai';

export class IntentClassifier {
  private genAI: GoogleGenerativeAI;
  private cache: Map<string, string>;

  constructor(apiKey: string) {
    this.genAI = new GoogleGenerativeAI(apiKey);
    this.cache = new Map();
  }

  async classifyIntent(query: string): Promise<string> {
    // Check cache first
    if (this.cache.has(query)) {
      return this.cache.get(query)!;
    }

    const prompt = `You are an intent classifier for a stock market analysis system.
Classify the following user query into one of these categories:
- portfolio: Portfolio analysis, recommendations
- market_analysis: Market trends, sector analysis
- news: Financial news, company news
- risk_assessment: Risk evaluation, diversification
- historical_analysis: Historical data, trend analysis
- general_query: General questions about stocks

User query: "${query}"

Respond with only the category name.`;

    const model = this.genAI.getGenerativeModel({ model: 'gemini-2.0-flash' });
    const result = await model.generateContent(prompt);
    const intent = result.response.text().trim().toLowerCase();

    // Cache the result
    this.cache.set(query, intent);
    if (this.cache.size > 1000) {
      const firstKey = this.cache.keys().next().value;
      if (firstKey) this.cache.delete(firstKey);
    }

    return intent;
  }
}
\end{lstlisting}

Энэ код нь Gemini AI-г ашиглан хэрэглэгчийн хүсэлтийг зургаан ангилалд хуваана. Cache механизм нь ижил асуултыг дахин боловсруулахгүй байх замаар гүйцэтгэлийг сайжруулдаг.

\subsection{Investment Agent: МХБ өгөгдөл ашиглах}

Investment Agent нь МХБ-ийн бодит өгөгдлийг prompt-д оруулж, бодит тоо, өгөгдөл дээр үндэслэн хариулт өгдөг. Доорх код нь энэ процессийг харуулж байна.

\begin{lstlisting}[language=JavaScript, caption=Investment Agent using MSE data]
// File: backend/investment-agent/src/index.ts

async function generateAIResponse(payload: any) {
  const { action, query } = payload;

  // Fetch MSE data from PostgreSQL
  const mseResult = await db.query(
    'SELECT symbol, name, last_price, change_percent, volume 
     FROM mse_companies 
     ORDER BY market_cap DESC LIMIT 20'
  );
  const mseData = mseResult.rows;

  // Build MSE data context
  const mseContext = mseData
    .map(c => `${c.symbol}: ${c.name}, Price: ${c.last_price} MNT, 
               Change: ${c.change_percent}%, Volume: ${c.volume}`)
    .join('\n');

  // Create prompt with real MSE data
  const prompt = `You are an investment analyst for the Mongolian Stock Exchange.

Current MSE Market Data:
${mseContext}

User Query: ${query}

Provide detailed analysis using the real data above. 
Include specific numbers, percentages, and trends.`;

  // Call Gemini AI
  const model = genAI.getGenerativeModel({ model: 'gemini-2.0-flash-exp' });
  const result = await model.generateContent(prompt);
  const response = result.response.text();

  return response;
}
\end{lstlisting}

Энэ код нь МХБ-ийн топ 20 компанийн өгөгдлийг татаж, prompt-д оруулж, Gemini AI-д илгээнэ. Үр дүн нь бодит тоо, өгөгдөл дээр үндэслэсэн дүн шинжилгээ болно.

\subsection{Knowledge Agent: Vector Search}

Knowledge Agent нь RAG систем ашиглан semantic search хийдэг. Доорх код нь vector embedding үүсгэж, cosine similarity ашиглан хамгийн холбогдолтой баримтуудыг олох логикийг харуулж байна.

\begin{lstlisting}[language=JavaScript, caption=Knowledge Agent RAG system]
// File: backend/knowledge-agent/src/index.ts

import { pipeline } from '@xenova/transformers';

async function performSemanticSearch(query: string) {
  // Load embedding model
  const embedder = await pipeline(
    'feature-extraction', 
    'Xenova/all-MiniLM-L6-v2'
  );

  // Generate query embedding
  const queryEmbedding = await embedder(query, { 
    pooling: 'mean', 
    normalize: true 
  });
  const queryVector = Array.from(queryEmbedding.data);

  // Search in PostgreSQL using pgvector
  const result = await db.query(
    `SELECT id, content, metadata, 
            1 - (embedding <=> $1::vector) AS similarity
     FROM knowledge_base
     WHERE 1 - (embedding <=> $1::vector) > 0.7
     ORDER BY similarity DESC
     LIMIT 5`,
    [JSON.stringify(queryVector)]
  );

  return result.rows.map(row => ({
    content: row.content,
    metadata: row.metadata,
    similarity: row.similarity
  }));
}
\end{lstlisting}

Энэ код нь Sentence-Transformers модел ашиглан query-г vector болгон хөрвүүлж, PostgreSQL-ийн pgvector extension ашиглан cosine similarity хайлт хийнэ.

\subsection{API Gateway: Kafka Integration}

API Gateway нь frontend-ээс хүсэлт хүлээн авч, Kafka руу дамжуулдаг. Доорх код нь энэ процессийг харуулж байна.

\begin{lstlisting}[language=JavaScript, caption=API Gateway Kafka producer]
// File: backend/api-gateway/src/routes/agent.routes.ts

router.post('/query', authenticateToken, async (req: Request, res: Response) => {
  try {
    const { query, type } = req.body;
    const userId = req.user.userId;
    const requestId = uuidv4();

    // Create Kafka message
    const message = {
      requestId,
      userId,
      query,
      type,
      timestamp: new Date().toISOString()
    };

    // Publish to Kafka
    await kafkaService.produce('user-requests', requestId, message);

    // Return immediately with requestId
    res.json({
      success: true,
      requestId,
      message: 'Query submitted successfully'
    });
  } catch (error) {
    logger.error('Failed to submit query', error);
    res.status(500).json({ error: 'Failed to submit query' });
  }
});
\end{lstlisting}

Энэ код нь хэрэглэгчийн асуултыг хүлээн авч, unique requestId үүсгэж, Kafka topic руу publish хийж, шууд хариулж байна. Агентууд асинхрон байдлаар боловсруулна.

\subsection{Frontend: SSE Streaming}

Frontend нь Server-Sent Events ашиглан агентын хариултыг бодит цагт хүлээн авдаг. Доорх код нь React компонент дахь SSE холболтыг харуулж байна.

\begin{lstlisting}[language=JavaScript, caption=Frontend SSE streaming]
// File: frontend/app/dashboard/chat/page.tsx

'use client';
import { useEffect, useState } from 'react';

export default function ChatPage() {
  const [messages, setMessages] = useState([]);
  const [streaming, setStreaming] = useState(false);

  const sendQuery = async (query: string) => {
    // Submit query
    const res = await fetch('/api/agent/query', {
      method: 'POST',
      headers: { 'Content-Type': 'application/json' },
      body: JSON.stringify({ query, type: 'market' })
    });
    const { requestId } = await res.json();

    // Connect to SSE stream
    setStreaming(true);
    const eventSource = new EventSource(
      `/api/agent/stream?requestId=${requestId}`
    );

    eventSource.onmessage = (event) => {
      const data = JSON.parse(event.data);
      if (data.type === 'response') {
        setMessages(prev => [...prev, {
          role: 'assistant',
          content: data.content
        }]);
      } else if (data.type === 'done') {
        eventSource.close();
        setStreaming(false);
      }
    };

    eventSource.onerror = () => {
      eventSource.close();
      setStreaming(false);
    };
  };

  return (
    <div className="chat-container">
      {/* Chat UI components */}
    </div>
  );
}
\end{lstlisting}

Энэ код нь асуулт илгээж, requestId авч, SSE холболт үүсгэж, агентын хариултыг бодит цагт хүлээн авч, UI-д харуулж байна.

\section{Бүлгийн дүгнэлт}

Энэхүү бүлэгт санал болгосон тархи мэт агентын микросервис загварыг бодитоор хэрэгжүүлсэн системийн дэлгэрэнгүй тайлбарыг өгсөн. Хэрэгжүүлэлт нь Монголын Хөрөнгийн Биржийн өгөгдөлд суурилсан санхүүгийн шинжилгээний вэб аппликейшн болж, онол, практикийн холбоог тодорхой харуулсан.

Системийн шаардлагууд нь функциональ болон техникийн хувьд дэлгэрэнгүй тодорхойлогдсон. Функциональ шаардлагын хувьд хэрэглэгчийн удирдлага, watchlist удирдлага, агентуудтай харилцах, санхүүгийн мэдээ, МХБ-ийн өгөгдөл зэрэг үндсэн функцүүд хамрагдсан. Техникийн шаардлагын хувьд агент бүр бие даасан микросервис болох, Kafka ашиглан асинхрон харилцах, Docker контейнерт ажиллах, 3-5 секундын хариу өгөх хугацаатай байх, Kafka message persistence хангах, ACID property дагах, Prometheus metrics дэмжих зэрэг шаардлагууд хангагдсан.

Технологийн стек нь орчин үеийн, production-ready технологиудаас бүрдсэн. Frontend талаас Next.js 14, React 18, TypeScript 5, Tailwind CSS, Shadcn/ui ашиглагдсан. Backend талаас Express.js, Node.js 20, TypeScript 5, Python 3.10, Apache Flink 1.18 ашиглагдсан. Message Broker болгон Apache Kafka 3.5, Zookeeper 3.8 сонгогдсон. Өгөгдлийн сан болгон PostgreSQL 16, Redis 7 ашиглагдсан. Хиймэл оюуны технологи болгон Google Gemini 2.0 Flash, Sentence-Transformers (all-MiniLM-L6-v2) ашиглагдсан. Гадаад API болгон Finnhub API, NewsAPI, Alpha Vantage зэрэг ашиглагдсан. Deployment технологи болгон Docker 24, Docker Compose 2.20 ашиглагдаж, үйлдвэрлэлийн орчинд Kubernetes ашиглах бэлтгэл хийгдсэн.

Хэрэглээний тохиолдлууд нь системийн бодит хэрэглээг тодорхой харуулсан. Хэрэглэгчийн бүртгэл ба нэвтрэлт нь персоналчилсан өглөөний мэдээлэл бүхий бүрэн процесстэй. Watchlist удирдлага нь үүсгэх, нэмэх, хасах, устгах үйлдлүүдийг Kafka events-тэй хослуулан хэрэгжүүлсэн. Агенттай харилцах нь Orchestrator-оос эхлэн Investment Agent хүртэлх бүх workflow-г асинхрон байдлаар харуулсан. Өдөр бүрийн мэдээний digest нь News Agent, Gemini AI, Email Service хамтын ажиллагааг харуулсан.

Хэрэгжүүлэлтийн явц нь системийн одоогийн байдлыг тодорхой харуулсан. Ойролцоогоор 70\% хэрэгжсэн байна. Бүрэн хэрэгжсэн хэсгүүд нь Infrastructure (Docker, Kafka, PostgreSQL), Агентууд (Orchestrator, Knowledge, Investment, News, PyFlink Planner үндсэн функцтэй), API Gateway (бүх endpoints), Frontend (authentication, dashboard, chat interface, watchlist), Database (МХБ-ийн өгөгдөл, seed data) зэргийг агуулна. Хэсэгчлэн хэрэгжсэн хэсгүүд нь PyFlink нарийн төвөгтэй функцүүд, Frontend-ийн МХБ market overview, stock detail pages, user settings, Monitoring ба analytics, Email notifications зэрэг. Хараахан эхлээгүй хэсгүүд нь Portfolio management, Risk assessment, Advanced analytics, Real-time market data, Machine learning features, Production deployment зэрэг ирээдүйн хөгжүүлэлтэд үлдсэн.

Кодын жишээнүүд нь системийн гол бүрэлдэхүүн хэсгүүдийн техникийн хэрэгжүүлэлтийг харуулсан. Orchestrator Agent-ийн Intent Classification нь Gemini AI ашиглан хэрэглэгчийн хүсэлтийг ангилах, cache механизм ашиглах логикийг харуулсан. Investment Agent нь МХБ-ийн бодит өгөгдлийг prompt-д оруулж, бодит тоо дээр үндэслэсэн дүн шинжилгээ өгөх процессийг харуулсан. Knowledge Agent нь RAG систем ашиглан vector embedding үүсгэж, cosine similarity хайлт хийх техникийг харуулсан. API Gateway нь Kafka producer болж, хүсэлтийг асинхрон дамжуулах механизмийг харуулсан. Frontend нь SSE streaming ашиглан агентын хариултыг бодит цагт хүлээн авах логикийг харуулсан.

Энэхүү бүлгээс харахад онолын бүлгүүдэд судалсан хиймэл оюуны инженерчлэл, агентын архитектур, микросервис, Event-Driven Architecture зэрэг зарчмууд нь бодит практикт амжилттай хэрэглэгдэж болох нь тодорхой болов. Хэрэгжүүлсэн систем нь санал болгосон тархи мэт агентын микросервис загварын үр ашигтай байдлыг харуулж, хиймэл оюун агентуудыг микросервис архитектурт event-driven байдлаар нэвтрүүлэх нь зөвхөн онолын түвшинд биш, практикт хэрэглэгдэх боломжтой технологи болохыг баталгаажуулсан. Систем нь 70\% хэрэгжсэн байгаа нь bachelor диплом ажлын хүрээнд хангалттай ахиц дэвшил бөгөөд үлдсэн 30\% нь ирээдүйн сайжруулалт, өргөжүүлэлтэд зориулагдсан.
