\chapter*{Нэр томьёоны тайлбар}
\addcontentsline{toc}{chapter}{Нэр томьёоны тайлбар}

\section*{Англи хэлнээс орчуулсан нэр томъёо}

\begin{longtable}{p{0.38\textwidth} p{0.55\textwidth}}
\caption{Англи хэлнээс орчуулсан нэр томъёо} \label{tab:terminology-english} \\
\textbf{Англи нэр} & \textbf{Монгол орчуулга} \\
\hline
\endfirsthead
\caption[]{Англи хэлнээс орчуулсан нэр томъёо (үргэлжлэл)} \\
\textbf{Англи нэр} & \textbf{Монгол орчуулга} \\
\hline
\endhead
AI Engineering & Хиймэл оюуны инженерчлэл \\
Foundation Model & Суурь модел \\
Language Model & Хэлний модел \\
Masked Language Model & Далдлагдсан хэлний модел \\
Large Language Model & Том хэлний модел \\
Labeled data & Тэмдэглэгдсэн өгөгдөл \\
Retrieval-Augmented Generation & Хайлтаар нэмэгдүүлсэн үүсгэлт \\
Retrieval algorithm & Хайлтын алгоритм \\
Prompt Engineering & Зааврын инженерчлэл \\
Machine Learning & Машин сургалт \\
Deep Learning & Гүн сургалт \\
Microservices Architecture & Микросервис архитектур \\
Event-Driven Architecture & Үйл явдлаар удирдагдах архитектур \\
Message Queue & Мессежийн дараалал \\
Vector Database & Векторын өгөгдлийн сан \\
Embedding & Векторчилсон хэсгийн төлөөлөл \\
Token & Токен (хэлний моделд тэмдэгт мөрийн таслалын нэгж) \\
Inference & Гаргалгаа (моделоос үр дүн гаргах үйл явц) \\
Fine-tuning & Нарийвчилсан сургалт \\
Pre-training & Урьдчилсан сургалт \\
Supervised Learning & Удирдлагатай сургалт \\
Self-supervised Learning & Өөрийгөө удирдсан сургалт \\
Reinforcement Learning & Бэхжүүлсэн сургалт \\
Temperature & Температур (моделийн бүтээлч чанарын параметр) \\
Hallucination & Төөрөгдөл (моделийн буруу мэдээлэл үүсгэх үзэгдэл) \\
Context Window & Контекстийн цонх \\
Agent & Агент (бие даан үйлдэл хийх систем) \\
Tool & Хэрэглүүр (хиймэл оюуны ашиглах хэрэглүүрүүд, ихэвчлэн API хэлбэрээр дамждаг.) \\
Orchestrator & Зохион байгуулагч \\
Consumer & Хэрэглэгч (мессеж хүлээн авагч) \\
Producer & Үйлдвэрлэгч (мессеж илгээгч) \\
Topic & Сэдэв (Kafka-гийн мессежийн ангилал) \\
Partition & Хэсэглэл \\
Stream Processing & Урсгал боловсруулалт \\
Latency & Хоцрогдол \\
Throughput & Дамжуулалт \\
Scalability & Өргөжих чадвар \\
Resilience & Уян хатан чанар байдал \\
Coupling & Хамаарал \\
Decoupling & Салангид байдал \\
Synchronous communication & Синхрон холбоо \\
Asynchronous communication & Асинхрон холбоо \\
Partition & Хэсэглэл \\
Topic & Сэдэв \\

\end{longtable}

\section*{Товчилсон үгс}

\begin{longtable}{p{0.12\textwidth} p{0.33\textwidth} p{0.48\textwidth}}
\caption{Товчилсон үгс} \label{tab:terminology-abbreviations} \\
\textbf{Товчлол} & \textbf{Нэр томьёо} & \textbf{Монгол тайлбар} \\
\hline
\endfirsthead
\caption[]{Товчилсон үгс (үргэлжлэл)} \\
\textbf{Товчлол} & \textbf{Нэр томьёо} & \textbf{Монгол тайлбар} \\
\hline
\endhead
LLM & Large Language Model & Том хэлний загвар \\
RAG & Retrieval-Augmented Generation & Хайлтаар нэмэгдүүлсэн үүсгэлт \\
API & Application Programming Interface & Програмчлалын интерфэйс \\
REST & Representational State Transfer & Төлөв байдлын төлөөлөлийн дамжуулалт \\
HTTP & Hypertext Transfer Protocol & Гипертекст дамжуулалтын протокол \\
JSON & JavaScript Object Notation & JavaScript объектын тэмдэглэгээ \\
SQL & Structured Query Language & Бүтэцлэгдсэн асуулгын хэл \\
BERT & Bidirectional Encentations foder Represrom Transformers & Transformer архитектурт суурилсан хэлний модел \\
GPT & Generative Pre-trained Transformer & Урьдчилан сурсан үүсгэгч Transformer \\
NLP & Natural Language Processing & Байгалийн хэлний боловсруулалт \\
ML & Machine Learning & Машин сургалт \\
MLOps & Machine Learning Operations & Машин сургалтын үйл ажиллагааны удирдлага \\
EDA & Event-Driven Architecture & Үйл явдлаар удирдагдах архитектур \\
SFT & Supervised Fine-tuning & Удирдлагатай нарийвчилсан сургалт \\
RLHF & Reinforcement Learning from Human Feedback & Хүний санал хүсэлтээр бэхжүүлсэн сургалт \\
DPO & Direct Preference Optimization & Шууд сонголтын оновчлол \\
TF-IDF & Term Frequency-Inverse Document Frequency & Нэр томьёоны давтамж ба баримтын урвуу давтамж \\
k-NN & k-Nearest Neighbors & k хамгийн ойр хөршүүд \\
ANN & Approximate Nearest Neighbors & Ойролцоо хамгийн ойр хөршүүд \\
FAISS & Facebook AI Similarity Search & Facebook-ийн хиймэл оюунт семантик хайлтын сан \\
gRPC & Google Remote Procedure Call & Google-ийн алсын процедур дуудлага \\
SOAP & Simple Object Access Protocol & Объект хандалтын энгийн протокол \\
MSE, МХБ & Mongolian Stock Exchange & Монголын Хөрөнгийн Бирж \\
CRUD & Create, Read, Update, Delete & Үүсгэх, унших, шинэчлэх, устгах үйлдлүүд \\
JWT & JSON Web Token & JSON форматаар илгээгддэг вэб токен \\
SSE & Server-Sent Events & Серверээс илгээгдэх үзэгдлийн урсгал \\
VaR & Value at Risk & Эрсдэлийн үнэлгээ \\
\end{longtable}

\section*{Техникийн нэр томъёо}

\begin{longtable}{p{0.25\textwidth} p{0.65\textwidth}}
\caption{Техникийн нэр томъёо} \label{tab:terminology-technical} \\
\textbf{Нэр томъёо} & \textbf{Тайлбар} \\
\hline
\endfirsthead
\caption[]{Техникийн нэр томъёо (үргэлжлэл)} \\
\textbf{Нэр томъёо} & \textbf{Тайлбар} \\
\hline
\endhead
Docker & Контейнержуулалтын платформ; сервис бүрийг тусдаа орчинд ажиллуулах боломж олгодог. \\
Apache Kafka & Салангид урсгалын платформ; өндөр дамжуулалттай мессежийн брокер. \\
Apache Flink & Урсгал боловсруулалтын фреймворк; бодит цагийн өгөгдөл боловсруулах хэрэглүүр. \\
PostgreSQL & Өгөгдлийн сангийн систем. \\
Redis & Санах ойд суурилсан өгөгдлийн сан; кэш болон богино хугацааны өгөгдөл хадгалах зориулалттай. \\
Node.js & JavaScript-ын ажиллах орчин; сервер талын хөгжүүлэлтэд ашиглагдана. \\
TypeScript & JavaScript-ийн супер багц хэл. \\
Python & Python Програмчлалын хэл; өгөгдөл боловсруулалт, AI хөгжүүлэлт зэрэгт ашиглагддаг. \\
Next.js & React суурилсан веб фреймворк. \\
Express.js & Node.js суурилсан веб фреймворк; RESTful API хөгжүүлэхэд өргөн ашиглагддаг. \\
Zookeeper & Салангид зохицуулалтын сервис; Kafka зэрэг системийн мета өгөгдөл, кластерийн төлвийг удирдана. \\
\end{longtable}

